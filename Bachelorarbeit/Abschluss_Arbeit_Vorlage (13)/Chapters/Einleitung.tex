\section{Einleitung}

\subsection{Ziele und Vorgehen}




Das Ziel dieser Arbeit ist die Entwicklung und Implementierung einer auf Quantenalgorithmen basierenden Bibliothek zur Kryptoanalyse aktueller Verschlüsselungsverfahren. 
Angesichts der Komplexität und des Fachwissens, das zur Entwicklung von Quantenalgorithmen erforderlich ist, 
konzentriert sich diese Arbeit auf die Implementierung und Anpassung prominenter Quantenalgorithmen aus der Literatur.
Die verwendeten Quantenalgorithmen können bestimmte Problemstellungen, 
die aufgrund ihrer Komplexität eine zentrale Bedeutung in modernen Verschlüsselungsverfahren haben, 
deutlich schneller lösen als die effizientesten klassischen Algorithmen.

In den zugrunde liegenden wissenschaftlichen Arbeiten werden diese Quantenalgorithmen als abstrakte oder konzeptuelle Algorithmen vorgestellt, 
ohne dass auf konkrete Implementierungsdetails eingegangen wird. 
Diese Arbeit beseitigt die Diskrepanz zwischen theoretischen Konzepten und praktischen Realisierungen, 
indem auf der Basis der abstrakten Quantenalgorithmen eine konkrete Implementierung entwickelt wird.
Anschließend werden die resultierenden Implementierungen der Quantenalgorithmen mit klassischen Algorithmen kombiniert, 
um die Problemstellungen, die für die Sicherheit der Verschlüsselungsverfahren entscheidend sind, effektiv lösen zu können. 

Diese Arbeit implementiert die Bibliothek auf der Abstraktionsebene von Quantenschaltkreisen.
Dazu wird das Open-Source-Softwareentwicklungskit Qiskit genutzt das auf der Programmiersprache Python basiert.

\subsection{Motivation}
In den letzten Jahren haben Fortschritte in der Forschung und Entwicklung von Quantencomputern neue Möglichkeiten für die praktische Untersuchung von Quantenalgorithmen ermöglicht.
Gegenwärtig ermöglichen sowohl Simulatoren von Quantencomputer als auch real existierende, wenn auch leistungsbegrenzte, Quantencomputer die Durchführung praktischer Tests.
Zur Zeit der ursprünglichen Konzeption der in dieser Arbeit verwendeten Quantenalgorithmen war eine praktische Erprobung entweder undenkbar oder nur durch umständliche Experimente mit stark vereinfachten Versuchen möglich.

Indem diese technologischen Möglichkeiten zur Ausführung und Erprobung der implementierten Quantenalgorithmen genutzt werden, 
eröffnet sich ein neuer Standpunkt, der die Betrachtung aus anderen Blickwinkeln erlaubt.


\subsection{Gliederung}
*Hier wird dann beschrieben in welchen Kapiteln der Leser welche Inhalte erwarten kann und wieso die Reihenfolge der Kapitel so gewählt wurden, inklusive der Struktur der Arbeit*


