\section{Shor-Algorithmus}
Im folgenden Kapitel wird der Shor-Algorithmus zum Faktorisieren von Zahlen beschrieben.
Der Inhalt bezieht sich auf den Zweck, die Funktionsweise und den Aufbau des Algorithmus.
Der Aufbau beinhaltet nicht die konkrete Implementierung der Bestandteile des Algorithmus.
Stattdessen werden die Details bezüglich der Implementierung im nächsten Kapitel behandelt.

\subsection{Zweck}
Der Shor-Algorithmus wurde mit dem spezifischen Ziel entwickelt, 
große zusammengesetzte Zahlen effizient auf Quantencomputern in ihre Primfaktoren zu faktorisieren. 
Im Gegensatz zu klassischen Faktorisierungsverfahren, die exponentielle Zeit erfordern~\cite{katz2023}, 
ermöglicht Shor's Ansatz die Faktorisierung in polynomialer Zeit, 
in Bezug auf die Anzahl an Bits der zu faktorisierenden Zahl~\cite{Shor_1997}. 
Dies stellt eine signifikante Beschleunigung gegenüber den besten bekannten klassischen Faktorisierungsverfahren dar.
Der Shor-Algorithmus bekräftigt die These, das Quantencomputer
bestimmte Probleme wesentlich schneller lösen können als ihre klassischen Gegenstücke.

\subsection{Funktionsweise} \label{Funktionsweise}
Der Shor-Algorithmus verwendet zwei Teilberechnungen, die zusammen die Faktorisierung berechnen.
Die erste Teilberechnung erfolgt mittels eines Quantenalgorithmus. 
Der zweite Teil basiert auf einem klassischen Algorithmus.
Im quantenmechanischen Teil des Shor-Algorithmus geht es um die Bestimmung der Ordnung in der multiplikativen Gruppe modulo \(N\).
Hierbei ist anzumerken, dass der Quantenalgorithmus nicht direkt die Primfaktoren der zu faktorisierenden Zahl berechnet.
Stattdessen wird die Eigenschaft ausgenutzt, 
dass das Problem der Faktorisierung äquivalent zu dem Problem der Ordnungsbestimmung ist~\cite{nielsen_chuang_2010}.
Daher impliziert eine effiziente Lösung für die Berechnung der Ordnung eine ebenso effiziente Methode zur Faktorisierung.
Der nachfolgende klassische Algorithmus verwendet die ermittelte Ordnung, 
um die Primfaktoren abzuleiten. 
Beide Teilberechnungen, sowohl die quantenmechanische als auch die klassische, 
führen ihre Berechnungen in polynomialer Zeit durch. 
Daher liegt die gesamte Laufzeit des Shor-Algorithmus ebenfalls in einer polynomialen Größenordnung.

\subsection{Ordnungsbestimmung}
Zu bestimmen sind die Primfaktoren der Zahl \(N\).
Zuerst wird ein \(a\) mit \(0 < a < N\) gewählt.
Falls der ungewöhnlichen Fall eintritt, dass \(a\) nicht teilerfremd zu \(N\) ist, entspricht \(a\) einem der Primfaktoren.
Anschließend wird die Ordnung beziehungsweise Periode \(p\) der Funktion \({f(x) = a^x \mod N}\) mit einem Quantenalgorithmus bestimmt:

Die Periode \(p\) beschreibt das kleinste ganzzahlige Element mit \({p > 0}\), für das gilt: \({f(p) = 1 \mod N}\).
\[
\begin{tabular}{l|llllll}
    x     &     0     &     1       &     2      &      3   &  4 &  5  \\ \hline
    \(7^x \mod 15\)    &      1     &        7     &       4     &     13     &  1 &  7 
\end{tabular} \longmapsto p = 4
\]

Im Wesentlichen handelt es sich bei der quantenmechanischen Berechnung des Shor-Algorithmus um die Quanten-Phase-Estimation.
Die Architektur dieser spezifischen Quanten-Phase-Estimation korrespondiert weitgehend mit der in Abschnitt~\ref{Quanten-Phase-Estimation} vorgestellten Struktur.
Hierbei ersetzen speziell für die gegebene Anwendung definierte \(U\)-Gatter die allgemeinen.

Für den konkreten Kontext der Periodenberechnung, realisieren die \(U\)-Gatter die Transformation:
\[U\ket{y} = \ket{ay \mod N}\] 
Die Ausführung der Quanten-Phase-Estimation erfordert die Erzeugung eines Eigenvektors der Transformation \(U\).
Da die Quanten-Phase-Estimation \(\varphi\) aus dem Eigenwert extrahiert, 
darf der Eigenvektor nicht den trivialen Eigenwert 1 besitzen.
Stattdessen ist es notwendig, dass die Periode der Transformation im Eigenwert enthalten ist.

Wie in~\cite*{nielsen_chuang_2010} gezeigt wird, gibt es zu \(U\) Eigenvektoren \(\ket{u_s}\), 
mit \(0 \leq s \leq p-1\): 
\[\ket{u_s} \equiv
\frac{1}{\sqrt{p}}
\sum_{k=0}^{p-1} e^{\frac{-2 \pi i s k}{p}\ket{a^k \mod N}} %ich soll die Formel mit der Summe sein
\]
Für \(a=7\) bei \(N=15\) entspricht \(r=4\).
Ein Eigenvektor zu \(s=1\) lautet dann:
\[
    \ket{u_1}_4 =
    \frac{1}{\sqrt{4}}(
        \ket{1}_4 + 
        e^{-\frac{2 \pi i}{4}}\ket{7}_4 + 
        e^{-\frac{4 \pi i}{4}}\ket{4}_4+ 
        e^{-\frac{6 \pi i}{4}}\ket{13}_4
    )
    \]
\[
    U\ket{u_1}_4 =
    \frac{1}{\sqrt{4}}(
        \ket{7}_4 + 
        e^{-\frac{2 \pi i}{4}}\ket{4}_4 + 
        e^{-\frac{4 \pi i}{4}}\ket{13}_4+ 
        e^{-\frac{6 \pi i}{4}}\ket{1}_4
    )
    \]
\[
    U\ket{u_1}_4 =
    e^{\frac{2 \pi i}{4}}
    \cdot
    \frac{1}{\sqrt{4}}(
        e^{-\frac{2 \pi i}{4}}\ket{7}_4 + 
        e^{-\frac{4 \pi i}{4}}\ket{4}_4 + 
        e^{-\frac{6 \pi i}{4}}\ket{13}_4+ 
        e^{-\frac{8 \pi i}{4}}\ket{1}_4
    )
    \]
\[
    U\ket{u_1}_4 =
    e^{\frac{2 \pi i}{4}}
    \cdot
    \frac{1}{\sqrt{4}}(
        e^{-\frac{8 \pi i}{4}}\ket{1}_4
        e^{-\frac{2 \pi i}{4}}\ket{7}_4 + 
        e^{-\frac{4 \pi i}{4}}\ket{4}_4 + 
        e^{-\frac{6 \pi i}{4}}\ket{13}_4 )
    =
    e^{\frac{2 \pi i}{4}} \cdot
    \ket{u_1}_4
    \]
Das kann verallgemeinert werden:
\[U\ket{u_s} = e^{\frac{2 \pi i s}{p}}\ket{u_s}\]

Wie man in der Definition von \(\ket{u_s}\) %ref zu Formel mit der Summe
sieht, 
benötigt die Initialisierung eines Eigenvektors \(\ket{u_s}\) mit einem konkreten \(s\) die Periode \(p\).
Man kann diese Problematik jedoch umgehen indem man anstelle eines einzelnen Eigenvektors \(\ket{u_s}\)
eine Superposition verwendet, die alle \(\ket{u_s}\) umfasst.
Die Superposition entspricht~\cite*{nielsen_chuang_2010}:
\[\frac{1}{\sqrt{p}} \sum_{s=0}^{r-1}\ket{u_s} = \ket{1}\] 
Also werden die Qubits, die für den Eigenvektor bestimmt sind, mit dem Zustand \(\ket{1}\) initialisiert.

Die Superposition der Eigenvektoren hat zur Folge, 
dass nach der inversen Quanten-Fourier-Transformation, also am Ende der Quanten-Phase-Estimation,
ebenfalls eine Superposition mit dem \(\varphi_s\) der Eigenwerte \(\lambda_{u_s}\) aller möglichen Eigenvektoren \(\ket{u_s}\) existiert.
Sei \(k\) die Anzahl an Kontroll-Qubits dann:
\[
    \frac{1}{\sqrt{p}} \sum_{s=0}^{p-1}\ket{2^k \cdot \frac{s}{p}}  = 
    \frac{1}{\sqrt{p}} (\ket{0} + \ket{2^k \cdot \frac{1}{p}} + \ket{2^k \cdot \frac{2}{p}} ... + \ket{2^k \cdot \frac{p-1}{p}})
    \]


Bei einer Messung wird also zufällig eines der \(\varphi_s \approx \frac{s}{p}\) gemessen.

Die Genauigkeit von \(\frac{s}{p}\) gegenüber dem \(\varphi_s\) hängt davon ab,
mit wie viele \(U^{2^x}\)-Gatter beziehungsweise mit wie viele Kontroll-Qubits,
die Quanten-Phase-Estimation ausgeführt wurde.

Mit einer ausreichenden Anzahl an verwendeten Kontroll-Qubits, 
können die Zustände vollkommen beschrieben werden und 
somit bei einer Messung die genauen \(\varphi_s\) gefunden.  

Falls nicht ausreichend Kontroll-Qubits vorhanden sind,
wird die Messung mit hoher Wahrscheinlichkeit, 
in den Zustand kollabieren der dem genauen Ergebnis am nächsten ist. 
Die Ergebnisse sind dann jedoch probabilistischer Natur, 
vergleichbar mit dem Beispiel aus dem Quanten-Phase-Estimation Kapitel, 
in Abbildung~\ref*{fig:3_qubit_qpe_measurment_uncertain}.

Anhand des gemessenen \(\frac{s}{p}\) erfolgt die Primfaktorzerlegung in einer klassischen Nachberechnung.

\subsection{Klassische Nachberechnung}
%Detailierter bezüglich Wahrscheinlichkeiten die Periode bzw Faktoren zu finden
Wie der Name dieses Abschnittes vermuten lässt, 
wird die Nachberechnung in der Regel mit einem klassischen Algorithmus durchgeführt.

Aus dem Messergebnis des Quanten-Phase-Estimation soll die Phase extrahiert werden.
Anhand der Messung sind die ersten \(k\) Bits von \(\varphi_s\) bekannt.
Dabei steht \(k\) für die Genauigkeit der Quanten-Phase-Estimation.
Die Genauigkeit \(k\) wird durch die Anzahl an Kontroll-Qubits festgelegt.
Sollten \(k\) Bits nicht vollständig ausreichen um \(\varphi_s\)  zu beschreiben, 
wird man bei der Messung eine Kommazahl erhalten die nah von \(\varphi_s\) liegt,
dieser aber nicht ganz entspricht.
Wendet man den Kettenbruch-Algorithmus auf das Messergebnis an,
wird von der Kommazahl aus, der nächste ganzzahlige Bruch gefunden.
Man kann zeigen, dass die Verwendung von \(k = 2n+1\) Kontrollqubits eine ausreichende Genauigkeit liefert,
so dass das Messergebnis unter Verwendung des Kettenbruch-Algorithmus zu einem Näherungsbruch \(\frac{s}{p}\) von \(\varphi_s\) führt~\cite*{nielsen_chuang_2010}.

Nichtsdestotrotz kann selbst bei der Verwendung von ausreichend Kontroll-Qubits ein \(\frac{s'}{p'}\) berechnet werden,
welches nicht die Periode \(p\) enthält.
Dies tritt auf wenn \(s\) und \(p\) einen gemeinsamen Teiler haben der die Kürzung von \(\frac{s}{p}\) auf \(\frac{s'}{p'}\) ermöglicht.

Findet man in der ersten Messung \(\frac{s'}{p'}\) und in einer zweiten \(\frac{s''}{p''}\), 
wobei \(s'\) und \(s''\) keine Faktoren teilen,
so kann aus dem kleinsten gemeinsamen Vielfachen von \(p'\) und \(p''\) \(p\) berechnen.

Man kann zeigen, 
dass bei \(2\log(N)\) Messungen, die Wahrscheinlichkeit sehr hoch ist, mindestens einmal ein \(\frac{s}{p}\)
zu messen, bei dem \(s\) und \(p\) teilerfremd sind~\cite*{nielsen_chuang_2010}.

Ob die korrekte Periode gefunden wurde, kann mit \(a^p = 1 \mod N\) geprüft werden.

Nach einige fehlgeschlagenen Versuchen(z.B. \(2\log(N)\)) wiederholt man die Suche mit einem anderen \(a\).

Sobald die korrekte Periode \(p\) gefunden wurde, 
können die Primfaktoren von \(N\)mit dem gemeinsamen Teiler(gcd) berechnet werden:
\[gcd(a^{\frac{p}{2}}-1, N), gcd(a^{\frac{p}{2}}+1, N)\]
Dies schlägt nur fehl falls \(r\) ungerade ist
oder falls \(a^{\frac{p}{2}} = -1 \mod N\) erfüllt~\cite*{Shor_1997}.
Die Wahrscheinlichkeit dass einer der beiden genannten Fälle eintritt beträgt \(1-\frac{1}{2^k}\), 
wobei \(k\) die Anzahl an unterschiedlicher Primfaktoren von \(N\) angibt~\cite*{Shor_1997}.
In einem solchen Fall, wiederholt man die Periodenberechnung mit einem anderen \(a\).




