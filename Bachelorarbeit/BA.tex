\documentclass[
  12pt,
  headsepline,
]{scrartcl}
\usepackage{graphicx}
\usepackage{physics}
\graphicspath{{./Bilder/}}
\usepackage[utf8]{inputenc}
\usepackage[T1]{fontenc}
\usepackage{amsmath,amsfonts,amssymb}
\usepackage{stmaryrd} 
\usepackage[ngerman]{babel}
\renewcommand{\listfigurename}{Abbildungsverzeichnis}
\usepackage{csquotes}
\usepackage{amsmath}
\usepackage{mathtools}
\usepackage{minted}
\usepackage{url}
\usepackage{hyperref}
\hypersetup{
  pdfauthor={},
  pdftitle={},
  pdfsubject={Bachelorarbeit},
  pdfkeywords={Bachelorarbeit}
}

\bibliographystyle{IEEEtran}
\usepackage[style=alphabetic]{biblatex}
\addbibresource{BA_Bib.bib}

% Bilder, Code-Blöcke, Farben, ...
\usepackage{float}
\usepackage{listings}
\lstset{
  showstringspaces=false,
  frame=lines,
  breaklines=true,
  language=Python,
  captionpos=b,
  prebreak={\Righttorque},
  basicstyle=\ttfamily
}
\usepackage{xcolor}

% Logo auf der Titelseite
\usepackage[absolute]{textpos}
\setlength{\TPHorizModule}{1mm}
\setlength{\TPVertModule}{\TPHorizModule}
\textblockorigin{0mm}{0mm}

\hyphenation{
  Ad-mi-nis-tra-tor
  Get-Re-quest
  Get-Re-sponse
  Get-Next-Request
  Get-Bulk-Request
  Pro-gram-mie-rung
  fast-ether-net hexa-de-zi-mal-en
}

\DeclareMathOperator{\Landau}{\mathcal{O}}
\DeclareMathOperator{\Tensor}{\bigotimes}
\DeclareMathOperator{\tensor}{\otimes}

\begin{document}

\begin{titlepage}

\thispagestyle{empty}%
\setlength{\oddsidemargin}{0cm}%
\enlargethispage{\baselineskip}
\begin{textblock}{20}(188,8)%
    \includegraphics[width=1.5cm]{fh_logo_rechts.png}%
 \end{textblock}%
\vspace*{-1.0cm}
\noindent
\LARGE\textbf{FH Aachen}\\
\Large\textbf{Fachbereich Elektrotechnik und Informationstechnik}\\
\large Prof. Dr. Marko Schuba
\vspace{2cm}
\begin{center}
	\LARGE\textbf{Bachelorarbeit}\\
	\vspace{1.75cm}
	\LARGE Implementierung einer Bibliothek von Quantenalgorithmen zur Kryptoanalyse\\
	\large
	\vspace{2.5cm}
	Eingereicht von:\\
	Simon Maximilian Kalytta\\
	Matrikelnummer: 3190683\\
	\vspace{1cm}
	Studienrichtung: Informatik\\
	\vspace{1cm}
	\today\\
	\vspace{1.5cm}
	Betreuerin: B.Sc. Janis König\\
	Prüfer: Prof. Dr. Marko Schuba\\
        \vspace{0.5cm}
        In Kooperation mit it.sec GmbH
\end{center}
\end{titlepage}

%Kurzfassung und Abstract gleich nur Deutsch und Englisch
\section*{Kurzfassung}
\textbf{Schlagwörter:} placeholder
%
\section*{Abstract}

\textbf{Keywords:} placeholder

\tableofcontents			% Inhaltsverzeichnis
%\listoftables				% Tabellenverzeichnis
\listoffigures				% Abbildungsverzeichnis
%\printnomenclature			% Abkürzungsverzeichnis

\section*{Abkürzungsverzeichnis}
\markright{Abkürzungsverzeichnis} 
\begin{acronym}[Abkürzungen]
	\acro{RSA}{Rivest Shamir Adleman}
	\acro{Qubit}{Quantenbit}
    \acro{IBM}{International Business Machine}
\end{acronym}
\input{Chapters/Glossar.tex}

\section{Einleitung}

\subsection{Ziele und Vorgehen}
Das Ziel dieser Arbeit ist die Entwicklung und Implementierung einer auf Quantenalgorithmen 
basierenden Bibliothek zur Kryptoanalyse aktueller Verschlüsselungsverfahren. 
Angesichts des spezialisierten Fachwissens, das für die Entwicklung von Quantenalgorithmen erforderlich ist, 
konzentriert sich diese Arbeit auf die Implementierung und Anpassung eines in der Literatur bereits vorgestellten Quantenalgorithmus.
Der ausgewählte Quantenalgorithmus ist in der Lage, eine spezifische Problemstellung, 
die aufgrund ihrer Komplexität eine zentrale Rolle in modernen Verschlüsselungsverfahren einnimmt, 
deutlich effizienter zu lösen als die besten klassischen Algorithmen.

In der zugrunde liegenden wissenschaftlichen Literatur wird dieser Quantenalgorithmus als abstraktes oder konzeptuelles Modell beschrieben, 
ohne dass auf konkrete Implementierungsdetails eingegangen wird. 
Diese Arbeit beseitigt die Diskrepanz zwischen theoretischen Konzepten und praktischen Realisierungen, 
indem auf Basis der abstrakten Konzepte eine konkrete Implementierung entwickelt wird.
Anschließend werden die resultierenden Implementierungen der Quantenalgorithmen mit klassischen Algorithmen kombiniert, 
um die Problemstellungen, die für die Sicherheit der Verschlüsselungsverfahren entscheidend sind, effektiv lösen zu können. 

Diese Arbeit implementiert die Bibliothek auf der Abstraktionsebene von Quantenschaltkreisen.
Dazu wird das Open-Source-Softwareentwicklungskit Qiskit genutzt, das auf der Programmiersprache Python basiert.

\subsection{Motivation}
In den letzten Jahren haben Fortschritte in der Forschung und Entwicklung von Quantencomputern neue Möglichkeiten für die praktische Untersuchung von Quantenalgorithmen ermöglicht.
Gegenwärtig ermöglichen sowohl Simulatoren von Quantencomputern als auch real existierende, wenn auch leistungsbegrenzte, Quantencomputer die Durchführung praktischer Tests.
Zur Zeit der ursprünglichen Konzeption der in dieser Arbeit verwendeten Quantenalgorithmen war eine praktische Erprobung entweder undenkbar oder nur durch umständliche Experimente mit stark vereinfachten Versuchen möglich.

Indem diese technologischen Möglichkeiten zur Ausführung und Erprobung der implementierten Quantenalgorithmen genutzt werden, 
eröffnet sich ein neuer Standpunkt, der die Betrachtung aus anderen Blickwinkeln erlaubt.


\subsection{Gliederung}
*Hier wird dann beschrieben in welchen Kapiteln der Leser welche Inhalte erwarten kann und wieso die Reihenfolge der Kapitel so gewählt wurden, inklusive der Struktur der Arbeit*



\section{Schwerpunkt}
\subsection{Entschlüsselungsfunktion für RSA-Verfahren} 
Der Fokus dieser Arbeit liegt auf der Implementierung einer Entschlüsselungsfunktion, die in der Lage ist, 
den privaten Schlüssel des RSA-Verfahrens aus dem zugehörigen öffentlichen Schlüssel abzuleiten.

Das RSA-Verfahren stellt ein sogenanntes asymmetrisches Kryptosystem dar.
Bei asymmetrischen Kryptosystemen kommt ein mathematisch verknüpftes Schlüsselpaar zum Einsatz. 
Dieses Schlüsselpaar besteht aus einem öffentlichen und einem privaten Schlüssel.
Wobei der privaten Schlüssel ausschließlich dem Eigentümer des Schlüsselpaares zugänglich sein sollte.
Mit dem privaten Schlüssel ist der Eigentümer in der Lage, Nachrichten zu signieren. 
Ein Nutzer kann unter der Verwendung des öffentlichen Schlüssels die Authentizität der signierten Nachricht überprüfen. 
Hingegen erlaubt der öffentliche Schlüssel die Verschlüsselung von Nachrichten, 
deren anschließende Entschlüsselung ausschließlich mit dem privaten Schlüssel durchführbar ist.
Eine Bedingung asymmetrischer Kryptosysteme ist, 
dass die Ableitung des privaten Schlüssels aus dem öffentlichen Schlüssel eine Herausforderung von erheblicher Komplexität darstellt~\cite{1055638}. 
Um dieser Anforderung gerecht zu werden, basieren asymmetrische Kryptosysteme auf mathematischen Problemstellungen, 
deren Lösung sogar unter Nutzung eines Computers von erheblicher Komplexität ist.

Die Sicherheit des kryptografischen Verfahren RSA beruht auf der Annahme,
dass die Faktorisierung eines Produkts bestehend aus zwei großen Primzahlen in keiner vertretbaren Zeit berechenbar ist.
Andernfalls wäre es möglich, aus dem öffentlichen Schlüssel die beiden Primfaktoren zu extrahieren, 
um anschließend damit den privaten Schlüssel zu bestimmen.
Daraus ergibt sich die Folgerung, dass, sofern die Faktorisierung großer Zahlen mit geringem Aufwand bewältigt werden kann, 
das RSA-Verfahren als kompromittiert angesehen werden muss~\cite{Cormen2009}.

Bislang ist kein klassischer Algorithmus bekannt, der die Primfaktorzerlegung effizient berechnen kann~\cite{Hoever2022Krypto}.
In diesem Zusammenhang bedeutet "`effizient"', dass die Laufzeit des Algorithmus in einem höchstens polynomialen Verhältnis zur Größe der Eingabezahl anwächst.

Im Kontext der Entschlüsselungsfunktion wird eine Funktion implementiert,
die in der Lage ist, die Primfaktorzerlegung effizient zu berechnen.
Dazu kombiniert die Entschlüsselungsfunktion den Shor-Algorithmus mit einem klassischen Algorithmus.
Der Shor-Algorithmus kann effektiv die Ordnung eines Elements in einer Gruppe bestimmen. 
Um auf bestimmte Sonderfälle bezüglich der Ergebnisse des Shor-Algorithmus zu reagieren, 
wird ein klassischer Algorithmus entwickelt, welcher diese Vorkommnisse behandelt. 
Dadurch wird die Qualität der Ergebnisse verbessert, was wiederum die Effizienz der Entschlüsselungsfunktion steigert.


Unter Einbeziehung der zuvor bestimmten Ordnung werden anschließend die Primfaktoren des öffentlichen Schlüssels berechnet.
In einem abschließenden Schritt wird der private Schlüssel auf der Grundlage der berechneten Primfaktoren ermittelt.




\section{Quantencomputer}
\subsection{Entwicklung} 
Die Grundidee der Quantencomputer findet ihren Ursprung im Jahr 1980, 
als Paul Benioff ein theoretisches Modell einer klassischen Turing-Maschine beschrieb, 
die den Gesetzen der Quantenmechanik unterlag.
Benioff demonstrierte, dass die Zustände einer Turing-Maschine in einem Quantensystem darstellbar sind. 
Er zeigte außerdem, dass klassische Operationen, die in einer Turing-Maschine ausgeführt werden, 
durch entsprechende Quantengatter auf einem Quantensystem äquivalent durchgeführt werden können~\cite{benioff1980}.

Kurz nach der Veröffentlichung von Benioffs Arbeit beschäftigte sich Richard Feynman im Jahr 1981 mit der Frage,
wie effizient ein klassischer Computer bei der Simulation physikalischer Prozesse ist.
Feynman befasste sich mit der Schwierigkeit, die Quantenmechanik mithilfe eines klassischen Computers zu simulieren.
Er stellte fest, dass die Anforderungen für die Simulation der Quantenmechanik auf einem klassischen Computer mit jedem zusätzlichen Quantenteilchen exponentiell ansteigen.
Die Begründung dafür liegt in der exponentiell wachsenden Menge an Zustandsinformationen,
die für die Beschreibung des Quantensystems benötigt werden.
Als Lösungsansatz nannte Feynman einen Quantencomputer, 
der selbst auf den Prinzipien der Quantenmechanik basiert und dadurch eine effiziente Simulation von Quantensystemen ermöglicht~\cite{Feynman1982}.

David Deutsch präsentierte im Jahr 1985 den ersten Quantenalgorithmus, 
der eine spezifische Fragestellung effizienter lösen konnte als alle bekannten klassischen Algorithmen. 
Allerdings bezog sich die betrachtete Fragestellung auf ein Problem von eher theoretischer Natur, 
in dem es darum ging, 
zu ermitteln, ob eine Funktion mit lediglich zwei möglichen Eingaben konstant oder balanciert ist.
Ein klassischer Algorithmus würde zur Beantwortung dieser Frage zwei Funktionsauswertungen benötigen. 
Hingegen benötigt der Quantenalgorithmus von Deutsch  
lediglich eine einzige Funktionsauswertung~\cite{deutsch1985}.

Im Jahr 1994 stellte Shor einen Quantenalgorithmus vor, der in der Lage ist, 
die Ordnung beziehungsweise Periode eines Elements in einer multiplikativen Gruppe eines Modulus mit nur polynomialem Aufwand zu ermitteln. 
Da trotz erheblicher Anstrengungen lediglich klassische Algorithmen mit exponentiellem Aufwand für diese Berechnung zur Verfügung stehen, 
stellte Shors Entdeckung eine Errungenschaft dar, die das allgemeine Interesse für Quantencomputing enorm anregte~\cite{Shor_1997}.

Zwei Jahre später stellte Lov~K.~Grover einen Quantenalgorithmus vor.
Genau wie der Quantenalgorithmus von Shor kann auch Grovers Quantenalgorithmus ein bestimmtes Problem effizienter lösen als jeglicher klassischer Algorithmus.
Dieser Quantenalgorithmus ermöglicht die Suche in einer unsortierten Datenbank mit einem Aufwand von \(\Landau(\sqrt N)\), 
während der beste klassische Suchalgorithmus diese Aufgabe in \(\Landau(N)\) bewältigt. 
Darüber hinaus zeigte Grover, 
dass es selbst unter vollständiger Ausnutzung der Prinzipien der Quantenmechanik nicht möglich ist, 
die unstrukturierte Suche in weniger als \(\Landau(\sqrt N)\) Schritten durchzuführen~\cite{grover1996fast}.

Die erste erfolgreiche physikalische Implementation eines Quantenalgorithmus wurde im Jahr 1998 durchgeführt.
In dem Versuch wurde der Quantenalgorithmus von Deutsch auf einem Zwei-Qubit-Quantencomputer implementiert.
Der verwendete Quantencomputer nutzte die Technologie der Kernspinresonanz, 
wobei die zwei Qubits durch Ausnutzung der Spin-Zustände von Atomkernen in spezifischen Molekülstrukturen realisiert wurden~\cite{Jones_1998}.

In der fortlaufenden Entwicklung der Quanteninformationstechnologie wurden verschiedene Arten von Quantencomputern konzeptioniert und realisiert.
Zu diesen zählen adiabatische Quantencomputer, Schaltkreis-basierte und topologische Quantencomputer.
Im Kontext des Quantencomputing wird der Begriff "`Quantencomputer"' oft auf universell einsetzbare Quantencomputer angewendet.
Die Eigenschaft "`universell"' wird in dem Zusammenhang gemäß DiVincenzos Kriterien definiert. 
Im Wesentlichen besagen DiVincenzos Kriterien, dass ein universeller Quantencomputer in der Lage sein muss, jede Quantenberechnung auszuführen, vorausgesetzt, er verfügt über ausreichende Ressourcen~\cite{DiVincenzo_2000}.
Schaltkreis-basierte und topologische Quantencomputer erfüllen DiVincenzos Kriterien und gelten somit als universelle Quantencomputer.
Hingegen sind adiabatische Quantencomputer nicht universell programmierbar und werden explizit zum Lösen von Optimierungsproblemen eingesetzt.

Das Unternehmen International Business Machines (IBM) präsentierte im Jahr 2019 den ersten kommerziell verfügbaren, Schaltkreis-basierten Quantencomputer.
Dieses System, bekannt als "`IBM Q System One"', verfügt über 20 Qubits.
Seit der Vorstellung des System One hat IBM die Qubit-Kapazität kontinuierlich erhöht. 
Aktuelle Systeme, wie die in der Abbildung~\ref{fig:IBM-Quantum-DevRoadmap2022} dargestellt sind, verfügen über bis zu 433 Qubits.
\begin{figure}[H]
    \centering
    \includegraphics[width=\columnwidth]{IBM-Quantum-DevRoadmap2022_Light.jpeg}
    \caption{IBM-Quantum-Roadmap~\cite{IBM_2023}}
    \label{fig:IBM-Quantum-DevRoadmap2022}
\end{figure}

In Anbetracht zukünftiger Entwicklungen haben Unternehmen wie Google, IBM und Microsoft umfangreiche Pläne zur Erweiterung ihrer Quantencomputing-Kapazitäten. 
Google plant beispielsweise die Errichtung eines Quanten-Campus, der bis 2030 über einen Quantencomputer mit einer Million Qubits verfügen soll~\cite{Google_2023}. 
Microsoft verfolgt ähnliche Ziele, wobei der konkrete Umfang ihrer Projekte bislang nicht öffentlich spezifiziert wurde.
Des Weiteren erwartet das Bundesamt für Sicherheit in der Informationstechnik die Existenz kryptografisch relevanter Quantencomputer zu Beginn der 2030er Jahre~\cite{BSI_KPMG_2023}. 

\section{Fundament}
\subsection{Literatur} 
\subsubsection*{Shor's Algorithmus}
Der Algorithmus wurde erstmals in der Publikation \textit{"`Algorithms for Quantum Computation: Discrete Logarithms and Factoring"'} von Peter W. Shor veröffentlicht.

Die Arbeit von Shor umfasste zwei Algorithmen.
Der erste Algorithmus ermöglicht die effiziente Berechnung der Primfaktorzerlegung, 
während der zweite Algorithmus die effiziente Berechnung des diskreten Logarithmus ermöglicht.
Aufgrund ihrer konzeptionellen Ähnlichkeiten werden beide Algorithmen häufig kollektiv als "`Shor's Algorithmus"' bezeichnet.

Die Quantenberechnungen beider Algorithmen basieren auf arithmetische Operationen in Restklassen und der Quanten-Fourier-Transformation.
Da Shor die Umsetzung von arithmetischen Operationen in Restklassen sowie die Quanten-Fourier-Transformation nicht explizit behandelt,
stützt sich die Implementierung auf den Ergebnissen weiterer Arbeiten. 
Diese untersuchen insbesondere effiziente Methoden zur Durchführung von arithmetischer Operationen in Restklassen innerhalb eines Quantenschaltkreises.
Einige dieser Arbeiten untersuchen die Quanten-Fourier-Transformation, 
da diese ein notwendiges Element für gewisse Berechnungen darstellt.

In der Publikation \textit{"`Quantum Networks for Elementary Arithmetic Operations"'} erklären Vlatko Vedral,  Adriano Barenco und Artur Ekert,
wie die modulare Exponentiation in einem Quantenschaltkreis berechnet werden kann. 

St\'{e}phane Beauregard baut auf den Erkenntnissen von Vedral, Barenco und Ekert auf und
verbessert den Quantenschaltkreis für die arithmetische Operation der modularen Exponentiation.
Hierfür ersetzt Beauregard einen Teil des ursprünglichen Quantenschaltkreis,
der in der modularen Exponentiation für die Berechnung der Addition genutzt wurde, 
durch einen effizienteren Quantenschaltkreis,
wie ihn Thomas G. Draper in \textit{"`Addition on a Quantum Computer"'} beschreibt.
Des weiteren verwendet Beauregard diese Optimierungen in seiner Arbeit \textit{"`Circuit for Shor’s algorithm using 2n+3 qubits"'},
um eine Realisierung von Shor's Algorithmus zur Faktorisierung zu beschreiben.

Andere Arbeiten, die sich mit der Implementierung von Shors's Algorithmus auseinandersetzen,
realisieren die modularen Exponentiation, 
indem klassische Schaltkreise identisch in den Quantenkontext übersetzt werden~\cite{Vedral_1996}.
Der Nachbau von klassischen Operationen ist aufgrund der unitären Natur von Quantenoperationen nicht die effizienteste Realisierung.
Des Weiteren meiden andere Arbeiten die Realisierung der modularen Exponentiation für allgemeine Eingaben~\cite{9376169,9686492}. 
Ohne modulare Exponentiation sind diese Varianten nicht in der Lage die Berechnung variabler Werte zu verarbeiten.
Stattdessen dienen diese ausschließlich der Demonstration der Funktionsweise von Shor's Algorithmus und
sind nur in der Lage, ausgewählte Eingaben zu verarbeiten.

Bei Recherchen wurde keine allgemeine Variante gefunden,
die weniger Qubits benötigt als die Variante aus der Arbeit \textit{"`Circuit for Shor’s algorithm using 2n+3 qubits"'},
deswegen bildet diese die Grundlage der Implementierung ab.

















\section{Grundlagen}
%*Im Folgenden Kapitel werden die Grundlagen erklärt die Notwenig sind um die 
% Aufgabe zu verstehen. Es werden nur auf die Eigenschaften eingegangen die Nötig sind um die fortlaufende
%Arbeit zu verstehen. "Irgendwie so etwas in die richtung "
\subsection{Qubits} 
"`Der Anfang enthält also beides, Sein und
Nichts; ist die Einheit von Sein und Nichts,
– oder ist Nichtsein, das zugleich Sein, und
Sein, das zugleich Nichtsein ist."'
- Georg Wilhelm Friedrich Hegel

Die Repräsentation und Speicherung von Information ist ein zentraler Aspekt aller Computer, 
unabhängig von der verwendeten Technologie oder Architektur. 
Bei klassischen Computern basiert diese Repräsentation auf dem Binärsystem. 
In der klassischen Repräsentation ist das Bit die kleinste Informationseinheit, 
die eine von zwei mögliche Zustände annehmen kann: 0 oder 1. 
Ein Bit hat zu jedem Zeitpunkt einen klar definierten Zustand. 
Mehrmaliges Auslesen eines Bits führt zu keiner Zustandsänderung, 
sofern keine Operationen zwischen den Auslesungen durchgeführt werden.

Quantencomputer hingegen funktionieren grundlegend anders. 
Sie stützen sich auf die Prinzipien der Quantenmechanik und verwenden anstelle von Bits Quatenbits beziehungsweise Qubits zur Informationsrepräsentation.
Ein einzelnes Qubit stellt die kleinstmögliche Informationseinheit dar über die ein Quantencomputer verfügt.
Um quantenmechanische Zustände von Qubits zu beschreiben wird die Dirac-Notation als mathematische Schreibweise genutzt~\cite{dirac_1939}.

Ein Qubit kann den Zustände 0 oder den Zustand 1 annehmen:
\begin{center}
\(\ket{0}\) oder entsprechend \(\ket{1}\)
\end{center}

Des weiteren kann ein Qubit auch beide der Basiszustände gleichzeitig einnehmen:
\begin{center}
\(\ket{\Psi} = \alpha\ket{0} + \beta\ket{1}\)
\end{center}
Dieses Phänomen ist eine der charakteristischen Eigenschaften von Qubits und wird als \textbf{Superposition} bezeichnet.
Dabei handelt es sich bei den Vorfaktoren \(\alpha\) und \(\beta\) um komplexe Zahlen für die gilt:
\begin{center}
\(\lvert\alpha\rvert^2 + \lvert\beta\rvert^2 = 1\)
\end{center}
Es gibt also unendlich viele Zustände, die ein Qubit in einer Superposition der beiden Basiszustände annehmen kann.

In der klassischen Welt findet man nur schwer eine Analogie zur Superposition. 
Man kann sich dies jedoch in etwa an einer Münze vorstellen~\cite{Hoever2023Münze}:

Ein klassisches Bit kann dabei entweder auf der Kopf- oder auf der Zahl-Seite liegen.
Ein Qubit hingegen ist eine Münze die auf der Kante um die eigene Achse rotiert.
Dabei geben die Vorfaktoren \(\alpha\) und \(\beta\) an,
wie die Münze zu der einen oder zu der anderen Seite tendiert.
Ohne die Münze zu beeinflussen ist es nicht möglich die Tendenz, 
also die Vorfaktoren, zu bestimmen.

Möchte man nun aber doch ein konkretes Ergebnis haben, 
muss man das Qubit lesen, 
beziehungsweise genauer gesagt messen.
Dabei wird die Superposition zerstört und das Qubit kollabiert in einen der beiden Basiszustände.

In der Analogie zu der Münze wird die Rotation der Münze gezielt gestoppt, sodass diese auf eine der beiden Seiten kippt.

Die Wahrscheinlichkeiten dafür werden durch die Vorfaktoren des Qubits bestimmt:
\begin{center}
\(\lvert\alpha\rvert^2\) für \(\ket{0}\), \(\lvert\beta\rvert^2\) für \(\ket{1}\)
\end{center}

Es ist also nicht möglich den Zustand eines Qubit während einer Berechnung nur "`anzuschauen"',
da ansonsten die Superposition zerstört wird und sich somit der Zustand des Qubits verändert.
Aus diesem Grund erfolgt die Messung in der Regel erst am Ende eines Quantenalgorithmus, um das endgültige Ergebnis zu ermitteln.

\subsection{Quantengatter}

\subsection{Qiskit}
\subsection{RSA}
%Funktionsweise
%Schwierigkeit Faktorisierung
%Laufzeit klassischer PC

\subsection{Quanten-Fourier-Transformation}
Die Quanten-Fourier-Transformation bildet einen wesentlichen Bestandteil des implementierten Quantenalgorithmus. 
Im folgen Abschnitt wird die allgemeine Anwendung der Quanten-Fourier-Transformation erklärt.
Darüber hinaus wird die Implementierung des Quantenschaltkreis anhand der Formel der Quanten-Fourier-Transformation hergeleitet.

Im Prinzip handelt es sich bei der Quanten-Fourier-Transformation um eine Transformation,
die Qubits von der Standardbasis (\(\ket{0}\), \(\ket{1}\)),
in die entsprechende Fourierbasis (\(\ket{+}\), \(\ket{-}\)) überführt~\cite{homeister2023quantum215}.
Bei dem Basiswechsel werden die Informationen des vorherigen Standardbasiszustandes in die Phase des neuen Zustandes übertragen~\cite{Ruiz-Perez2017}.
Anschließend können in der Fourierbasis Rechnungen durchgeführt werden die im Grunde durch Manipulationen der Phase realisiert werden.
Mit diesem Ansatz ist es möglich arithmetische Operationen, wie beispielsweise 
die Addition, effizienter zu berechnen~\cite{draper2000addition,Ruiz-Perez2017}.

Unter Verwendung der inversen Quanten-Fourier-Transformation, 
die als Rücktransformation dient, 
ist es möglich wieder zurück in die Standardbasis zu transformieren.
Dies bewirkt, die Extraktion der Information aus der Phase in einen messbaren Zustand. 

Die Quanten-Fourier-Transformation ist für ein \(N\) = \(2^n\) mit \(n\) Qubits
für die Basisvektoren \(\ket{x}, x = 0,...,N-1\) wie folgt definiert~\cite{Hoever2023QC}:
\[QFT_{N}\ket{x}_{n} = \frac{1}{\sqrt{N}}\sum_{y=0}^{N-1}  e^{\frac{2 \pi i x y}{N}}\ket{y}_{n}\]
Anhand dieser Definition kann der Quantenschaltkreis nicht direkt hergeleitet werden.
Stattdessen musst die Formel umgeformt werden.

Die folgende Herleitung stammt aus dem Textbuch \textit{Quantum Computation and Quantum Information} Seite 218~\cite{nielsen_chuang_2010}.
Die Herleitung wird der Übersicht halber um Zwischenschritte ergänzt:

Indem \(y\) in der ersten Formel in die Binärschreibweise übertragen wird, erhält man:
%Reihnfolge der Binärschreibweise einheitlich machen
% 
%\begin{align}
%    % -> & <-    &     ->&<-
%    122 &= 3     &   abc &= foo\\
%      x &= y     &
%      a &= 700   &
%\end{align}
%
%\begin{align}
% foo &= bar \\
%     &= baz \\
%     &= woop
%\end{align}
%
\[ y = \sum_{k=1}^{n}2^{n-k} y_{n-k+1}\]  
\begin{align}
  QFT_{N}\ket{x}_{n} &=
    \frac{1}{\sqrt{N}}
    \sum_{y_n=0}^{1} ...
    \sum_{y_1=0}^{1} e^{\frac{2 \pi i x \sum_{k=1}^{n}2^{n-k} y_{n-k+1}}{N}}
    \ket{y_n ... y_{2}y_{1}}
\end{align}
Mit \(N\) = \(2^n\) kann der Bruch im Exponenten gekürzt werden:
\[QFT_{N}\ket{x}_{n} = \frac{1}{\sqrt{N}}\sum_{y_n=0}^{1}...\sum_{y_1=0}^{1}  e^{\frac{2 \pi i x \sum_{k=1}^{n}2^{n-k} y_{n-k+1}}{2^n}}\ket{y_n ... y_{2}y_{1}}\]
\[QFT_{N}\ket{x}_{n} = \frac{1}{\sqrt{N}}\sum_{y_n=0}^{1}...\sum_{y_1=0}^{1}  e^{2 \pi i x \sum_{k=1}^{n}2^{-k} y_{n-k+1}}\ket{y_n ... y_{2}y_{1}}\]
Anschließend kann der Ausdruck \(2^{-k}\) zu \(\frac{1}{2^k}\) umgeformt werden: 
\[QFT_{N}\ket{x}_{n} = \frac{1}{\sqrt{N}}\sum_{y_n=0}^{1}...\sum_{y_1=0}^{1}  e^{\frac{2 \pi i x \sum_{k=1}^{n} y_{n-k+1}}{2^k}}\ket{y_n ... y_{2}y_{1}}\]
Die Summe im Exponenten der Basis \(e\) kann als Produkt umgeschrieben werden.
Anstatt dem Produktzeichen \(\prod\) wird das Tensorprodukt \(\bigotimes\) verwendet, da es sich um Qubits handelt:
\[QFT_{N}\ket{x}_{n} = \frac{1}{\sqrt{N}}\sum_{y_n=0}^{1}...\sum_{y_1=0}^{1} \bigotimes_{k=1}^n{ e^{\frac{2 \pi i x y_k}{2^k}}}\ket{y_{k}}\]
Der Ausdruck kann weiter vereinfacht werden indem das Tensorprodukt vorgezogen wird:
\[QFT_{N}\ket{x}_{n} = \frac{1}{\sqrt{N}}\bigotimes_{k=1}^n [  \sum_{y_k=0}^{1}{ e^{\frac{2 \pi i x y_k}{2^k}}}\ket{y_{k}}]\]
\[QFT_{N}\ket{x}_{n} = \frac{1}{\sqrt{N}}\bigotimes_{k=1}^n [  \ket{0} + { e^{\frac{2 \pi i x}{2^k}}}\ket{1}]\] 
Schreibt man das Tensorprodukt voll aus und notiert \(x\) in Binärschreibweise, erhält man:
\begin{align*}
\frac{1}{\sqrt{N}}\Bigg(
  \ket{0} + { e^{\frac{2 \pi i (2^{n-1}x_n+...+2^1x_2+2^0x_1)}{2^1}}}\ket{1}
  \Bigg)
  &\otimes
  \ket{0} + { e^{\frac{2 \pi i (2^{n-1}x_n+...+2^1x_2+2^0x_1)}{2^2}}}\ket{1} \\
  &\vdotswithin{\otimes} \\
  &\otimes
  \ket{0} + { e^{\frac{2 \pi i (2^{n-1}x_n+...+2^1x_2+2^0x_1)}{2^n}}}\ket{1}
\end{align*}
Die komplexe Exponentialfunktion ergibt für eine natürliche Zahl \(k\): \(e^{2\pi i k} = 1\).
Mit dieser Eigenschaft kann man beispielsweise die Phasenverschiebung des ersten Tensors vereinfachen:
\begin{align*} 
  e^{\frac{2 \pi i (2^{n-1}x_n+ \dotsb +2^1x_2+2^0x_1)}{2^1}}
  &\equiv
  e^{\frac{2 \pi i (2^{n-1}x_n)}{2^1}} \dotsb e^{\frac{2 \pi i (2^1x_2)}{2^1}} e^{\frac{2 \pi i (2^0x_1)}{2^1}} \\
  &\equiv 
  e^{{2 \pi i (2^{n-2}x_n)}} \dotsb e^{{2 \pi i (2^0x_2)}} e^{{2 \pi i (2^{-1}x_1)}}
\end{align*}
Dabei ergibt nur der Term \(e^{{2 \pi i (2^{-1}x_1)}} \neq 1 \)  
und verursacht somit eine relevante Phasenverschiebung.

Abschließend lässt sich das gesamte Tensorprodukt vereinfachen:
\[QFT_{N}\ket{x}_{n} = \frac{1}{\sqrt{N}}(\ket{0} + { e^{\frac{2 \pi i (2^0x_1)}{2^1}}}\ket{1}) \bigotimes
( \ket{0} + { e^{\frac{2 \pi i (2^1x_2+2^0x_1)}{2^2}}}\ket{1})...
( \ket{0} + { e^{\frac{2 \pi i (2^{n-1}x_n+ ... +2^1x_2+2^0x_1)}{2^n}}}\ket{1})\]

Ein einzelner Tensor repräsentiert die Wirkung der Schaltung auf ein einzelnes Qubit.
Somit sind die Phasenverschiebungen erkenntlich die auf ein Qubit wirken.
Ausserdem verdeutlicht die Binärschreibweise, dass die angewendete Phasenverschiebung vom Zustand anderer Qubits abhängt.

Für die Implementierung der Quanten-Fourier-Transformation sind nur die Terme relevant
welche eine Phasenverschiebung von \(\neq 1\) bewirken.
Dies kann man mit folgender Formel beschreiben:
\[QFT_{N}\ket{x}_{n} = \frac{1}{\sqrt{N}}\bigotimes_{k=1}^n [  \ket{0} + { e^{\frac{2 \pi i \sum_{b=0}^{k-1}2^b x_{b+1} }{2^k}}}\ket{1}]\] 
Im weiteren wird die Formel verwendet um ein Quantenschaltkreis der Quanten-Fourier-Transformation für drei Qubits zu implementieren:
\[QFT_{8}\ket{x}_{3} = \frac{1}{\sqrt{8}} [ (\ket{0} + { e^{\frac{2 \pi i (2^0x_1)}{2^1}}}\ket{1} ) \bigotimes
( \ket{0} + { e^{\frac{2 \pi i (2^1x_2+2^0x_1)}{2^2}}}\ket{1} ) \bigotimes
( \ket{0} + { e^{\frac{2 \pi i (2^{2}x_3 +2^1x_2+2^0x_1)}{2^3}}}\ket{1} ) ] \]
Man kann die Phasenverschiebung die durch den Zustand eines einzelnen Qubits erzeugt wird verdeutlichen,
indem man die Addition im Exponenten zu einer Multiplikation der gleichen Basen umformt:
\[ = \frac{1}{\sqrt{8}} [ (\ket{0} + { e^{\frac{2 \pi i (2^0x_1)}{2^1}}}\ket{1}) \bigotimes
( \ket{0} + { e^{\frac{2 \pi i (2^1x_2)}{2^2}} e^{\frac{2 \pi i (2^0x_1)}{2^2}} }\ket{1} ) \bigotimes
( \ket{0} + { e^{\frac{2 \pi i (2^{2}x_3)}{2^3}} e^{\frac{2 \pi i (2^1x_2)}{2^3}} e^{\frac{2 \pi i (2^0x_1)}{2^3}}  }\ket{1} ) ] \]
Die Phasenverschiebung wird eindeutiger indem man die Brüche kürzt:
\[ = \frac{1}{\sqrt{8}} [ ( \ket{0} + e^{\pi i x_1}\ket{1} ) \bigotimes
( \ket{0} + { e^{\pi i x_2} e^{ \frac{ \pi i x_1}{2}} }\ket{1} ) \bigotimes
( \ket{0} + { e^{\pi i x_3} e^{\frac{\pi i x_2}{2}} e^{ \frac{ \pi i x_1}{4}} }\ket{1} ) ] \]
In einer abschließenden Umformung lässt sich die \(\frac{1}{\sqrt8}\) aufteilen.
Dadurch erinnern die einzelnen Tensoren,
beziehungsweise Qubits an die Form die bei eine Hadamard-Transformation entsteht:
\[ = \frac{1}{\sqrt{2}}( \ket{0} + e^{\pi i x_1}\ket{1} ) \bigotimes
\frac{1}{\sqrt{2}}( \ket{0} + { e^{\pi i x_2} e^{ \frac{ \pi i x_1}{2}} }\ket{1} ) \bigotimes
\frac{1}{\sqrt{2}}( \ket{0} + { e^{\pi i x_3} e^{\frac{\pi i x_2}{2}} e^{ \frac{ \pi i x_1}{4}} }\ket{1} )  \]
Der Ausdruck \(e^{\pi i x_k}\) ist in jedem der einzelnen Qubits vorhanden.
Des weiteren ist an dem \(x_k\) erkennbar das die angewendete Phasenverschiebung von dem Zustand des Qubits abhängig ist,
auf welches die Verschiebung auch angewendet wird.
Konkret bedeutet dass, das auf jedes Qubit mit dem Zustand \(\ket{x_k} = \ket{1}\) eine Phasenverschiebung von \(e^{\pi i}\) wirkt.
Anhand des ersten Qubits, beziehungsweise Tensor, würde das bedeuten, dass der Zustand bei \(x_1 = 0\)
zu \(\ket{0} + e^{\pi i 0} \ket{1} \equiv \ket{0} + \ket{1}\) wird.
Bei \(x_1 = 1\) würde man \(\ket{0} + e^{\pi i 1} \ket{1}\) erhalten,
was \(\ket{0} - \ket{1}\) entspricht.
Aufgrund des Vorfaktors von \(\frac{1}{\sqrt{2}}\) entsprächen beide Fälle also der Hadamard-Transformation.
Da auch alle anderen Tensoren den Ausdruck \(e^{\pi i x_k}\) und den Vorfaktor \(\frac{1}{\sqrt{2}}\) beinhalten, wirkt auf jedes Qubit ein Hadamard-Gatter.

Die weiteren Tensoren des Tensorproduktes beinhalten zunehmend mehr Ausdrücke die für unterschiedliche Phasenverschiebungen sorgen.
Die Phasenverschiebung von einem einzelnen Ausdruck kann mit einem Phasen-Gatter realisiert werden.
Das Phasen-gatter entspricht einer Rotation mit \(P(\lambda) = 
\begin{pmatrix}
  1 & 0 \\
  0 & e^{i \lambda}
\end{pmatrix}\).

Beispielsweise wirkt auf dem zweiten Qubit noch ein Phase-Gatter mit \(P(\frac{\pi i}{2} )\).
Diese Phasenverschiebung soll aber nur angewendet werden wenn \(x_1 = 1\) ist.
Deswegen wird das Phase-Gatter durch den Zustand des ersten Qubits kontrolliert und
mit einem kontrollierten-Phase-Gatter realisiert.
Das gleiche Prinzip gilt für alle weiteren Qubits.
Anschließend erhält man einen Quantenschaltkreis wie in Abbildung~\ref{fig:qft}.

Wie im Quantenschaltkreis in Abbildung~\ref{fig:qft} erkennbar, 
spiegelt die Quanten-Fourier-Transformation die Reihenfolge der Qubits~\cite{Hoever2023QC}.
Um die Reihenfolge wiederherzustellen werden am Ende der Quanten-Fourier-Transformation Swap-Operationen verwendet.
\begin{figure}
\caption{3-Qubit QFT ohne Swaps}
\label{fig:qft}
%\includegraphics[width=\columnwidth]{qft.PNG}
\centering
\end{figure}

Anhand der Implementierung wird ersichtlich,
dass die Quanten-Fourier-Transformation unitär ist. 
Diese Eigenschaft ergibt sich aus der Tatsache, 
dass der zugehörige Quantenschaltkreis ausschließlich unter Verwendung von unitären Gattern realisierbar ist.

Wie bereits oben erwähnt, 
wird für die Rücktransformation aus der Fourierbasis in die Standardbasis 
die inversen Quanten-Fourier-Transformation angewendet.
Um einen Quantenschaltkreis aus unitären Gattern zu invertieren, 
wird die inverse der verwendeten Gatter
in umgekehrter Reihenfolge der originalen Schaltung angewendet.
Die Swap Operationen stehen somit bei der inversen Quanten-Fourier-Transformation am Anfang.
In Abbildung~\ref{fig:iqft} ist beispielhaft die inverse Quanten-Fourier-Transformation für drei Qubits abgebildet.
\begin{figure}
\caption{3-Qubit inverse QFT ohne Swaps}
\label{fig:iqft}
%\includegraphics[width=\columnwidth]{iqft.PNG}
\centering
\end{figure}

\subsection{Quanten-Phase-Estimation} \label{Quanten-Phase-Estimation}

Im nachfolgenden Abschnitt wird die Anwendung und Funktionsweise des Quantum-Phase-Estimation Quantenalgorithmus erläutert. 
Die Quantum-Phase-Estimation ist ein Bestandteil einiger fortgeschrittener Quantenalgorithmen und eine integrale Komponente des spezifischen Quantenalgorithmus, 
der in dieser Arbeit implementiert wird. 
Genauer gesagt basiert der implementierte Algorithmus auf den Prinzipien der Quantum-Phase-Estimation und verwendet die selbe Methodik für einen spezialisierten Kontext.

Schwerpunktmäßig konzentriert sich die Erklärung primär auf das Verständnis der Funktionsweise der Quantum-Phase-Estimation und nimmt an, dass einige Voraussetzungen gegeben sind. 
Diese Voraussetzungen hängen vom spezifischen Kontext ab, 
in dem die Quantum-Phase-Estimation angewendet wird. 
Im weiteren Verlauf der Arbeit werden diese Voraussetzungen im Hinblick auf den Anwendungsfall des implementierten Quantenalgorithmus konkretisiert.

Voraussetzungen ist, dass ein Eigenvektor \(\ket{x}_n\) von einer unitäre Transformation \(U^{n\times n}\) bekannt ist.
Wendet man die Transformation \(U^{n\times n}\) auf \(\ket{x}_n\) an, 
so gilt: \(U^{n\times n}\ket{x}_n=\lambda_x\ket{x}_n\)~\cite{nielsen_chuang_2010}.
Dabei erhält man, abhängig vom gewählten Eigenvektor \(\ket{x}_n\), einen der Eigenwert \(\lambda_x\) von \(U^{n\times n}\).
Ein Eigenwert \(\lambda_x\) besitzt die Form eines Phasenfaktors: \(e^{2\pi i \varphi}\)
mit \(0 \leq \varphi < 1\).
Im Prinzip wird durch die unitäre Transformation also eine globale Phasenverschiebung auf den Eigenvektor angewendet.

Wie bereits im Kapitel zu den Grundlagen gezeigt wurde, 
ist es nicht möglich eine globale Phase durch eine gewöhnliche Messung der Qubits zu bestimmen. 
Das liegt daran, 
dass eine globale Phase die Amplituden eines Qubits nicht verändert und somit die Wahrscheinlichkeiten der Messergebnisse unverändert bleiben.
Stattdessen muss man die Qubits manipulieren, so dass die globale Phase doch Einfluss auf die Amplituden hat.

Der Quanten-Phase-Estimation Quantenalgorithmus ist in der Lage, 
den Eigenwert aus \(U^{n\times n}\ket{x}_n=\lambda_x\ket{x}_n\), also die Phasenverschiebungen, repräsentiert durch \(\lambda_x = e^{2\pi i \varphi}\),
auf die Amplitude eines anderen Qubits zu verschieben.
Um \(\lambda_x\) auf ein anderes Qubit zu übertragen wird der Effekt des \textbf{Phase-Kickback} genutzt.
Anschließend wird der Wert \(\varphi\) des Eigenwertes durch die inverse Quanten-Fourier-Transformation in einen messbaren Zustand überführt.

Der Phase-Kickback tritt auf,
wenn eine unitäre Transformation \(U^{n\times n}\) kontrolliert durch ein Qubit \(\ket{y}_1\) in Superposition, 
auf einen Eigenvektor \(\ket{x}_n\) von \(U^{n\times n}\) anwendet wirkt.
Dabei wird der Eigenwert, beziehungsweise \(\lambda_x\), auf den \(\ket{1}\)-Anteil von \(\ket{y}_1\)
übertragen.

Sei: \(\ket{y}_1 \equiv \alpha\ket{0} + \beta\ket{1}\) mit \(\alpha,\beta \neq 0\), 
dann:

\(CU^{(n+1)\times (n+1)}(\ket{y}_1\bigotimes\ket{x}_n)=(\alpha\ket{0} + \lambda_x\beta\ket{1})\bigotimes\ket{x}_n\).

Es folgt ein Beispiel welches den Effekt verdeutlicht:
Beachtet werden zwei Qubits im Zustand \(\ket{+}_1 \bigotimes \ket{-}_1\) auf die ein kontrolliertes X-Gatter angewendet wird.
Dabei ist \(\ket{-}_1\) der Eigenvektor einer X-Transformation mit zugehörigen Eigenwert \(-\), also \(X^{1\times 1}\ket{-}_1=-\ket{-}_1\).

Auf das zweite Qubit \(\ket{-}_1\) wirkt ein \(CX^{2\times 2}\) Gatter welches durch das erste Qubit \(\ket{+}_1\) kontrolliert wird:
\[
  CX^{2\times 2}( \ket{+}_1\ \bigotimes \ket{-}_1) =
  \begin{pmatrix}
    1 & 0 & 0 & 0\\
    0 & 1 & 0 & 0\\
    0 & 0 & 0 & 1\\
    0 & 0 & 1 & 0
  \end{pmatrix}
  \cdot
  \frac{1}{{2}}
  \begin{pmatrix}
    1 \\
    -1 \\
    1 \\
    -1 
  \end{pmatrix}
  =
  \frac{1}{{2}}
  \begin{pmatrix}
    1 \\
    -1 \\
    -1 \\
    1 
  \end{pmatrix}
  \]
  \[
  =
  \frac{1}{{\sqrt{2}}}
  \begin{pmatrix}
    1 \\
    -1 
  \end{pmatrix}
  \bigotimes
  \frac{1}{{\sqrt{2}}}
  \begin{pmatrix}
    1 \\
    -1 
  \end{pmatrix}
  =
  \ket{-}_1 \bigotimes \ket{-}_1
  \]
Mithilfe des Phase-Kickback kann man also den Eigenwert einer unitären Transformation in die Phase eines Kontrollqubits verschieben.
Der Vorteil davon ist, dass diese Phasenverschiebung nur den Vorfaktor von \(\ket{1}\) des Kontrollqubits betrifft und keine globale Phase darstellt.

Der Aufbau eines Quanten-Phase-Estimation Quantenschaltung sieht beispielsweise wie folgt aus:

Beachtet wird die unitäre Transformation eines Phase-Gatter(P) mit einer variablen Phasenverschiebung von 
\(P(2 i \pi \varphi ) = 
\begin{pmatrix}
  1 & 0\\
  0 & e^{2 i \pi \varphi}
\end{pmatrix}\).

Ein zugehöriger Eigenvektor dieser Transformation ist \(\ket{1}_1\) den \(P(2 i \pi \varphi)\ket{1}_1 = e^{2 i \pi \varphi} \ket{1}_1\).

Um den Effekt das Phase-Kickback nutzen zu können, muss sich das Kontrollqubit in einer Superposition befinden.
Dafür wird das Kontrollqubit im Zustand \(\ket{0}_1\) initialisiert und
anschließend mit einem Hadamard-Gatter(H) in die gleichmäßige Superposition \(\ket{+}_1\) versetzt.
\[\ket{0}_1 \bigotimes \ket{1}_1 
\underrightarrow{H^{\bigotimes 1}}
 \ket{+}_1 \bigotimes \ket{1}_1
=
\frac{1}{\sqrt{2}}
\begin{pmatrix}
  1 \\
  1
 \end{pmatrix}
 \bigotimes
 \begin{pmatrix}
  0 \\
  1 
 \end{pmatrix}
 =
 \frac{1}{\sqrt{2}}
 \begin{pmatrix}
  0 \\
  1 \\
  0 \\
  1
\end{pmatrix}
 \]
Über das kontrollierte Phase-Gatter wird der Eigenwert auf das Kontrollqubit verschoben und
befindet sich deswegen nicht mehr im Zustand \(\ket{+}_1\) :
\[
  \frac{1}{\sqrt{2}}
  \begin{pmatrix}
   0 \\
   1 \\
   0 \\
   1
  \end{pmatrix}
  \underrightarrow{CP^{2\times 2}(2 i \pi \varphi)}
  \begin{pmatrix}
    1 & 0 & 0 & 0\\
    0 & 1 & 0 & 0\\
    0 & 0 & 1 & 0\\
    0 & 0 & 0 & e^{2 i \pi \varphi}
  \end{pmatrix}
  \cdot
  \frac{1}{\sqrt{2}}
  \begin{pmatrix}
   0 \\
   1 \\
   0 \\
   1
  \end{pmatrix}
  =
  \frac{1}{\sqrt{2}}
  \begin{pmatrix}
    0 \\
    1 \\
    0 \\
    e^{2 i \pi \varphi}
  \end{pmatrix}
  =
  \frac{1}{\sqrt{2}}
  \begin{pmatrix}
    1 \\
    e^{2 i \pi \varphi}
   \end{pmatrix}
   \bigotimes
   \begin{pmatrix}
    0 \\
    1 
   \end{pmatrix}
  \]
Anschließend wird auf die Kontrollqubits die inverse Quanten-Fourier-Transformation angewendet.
Die inverse Quanten-Fourier-Transformation sorgt dafür, 
dass der Eigenwert die Amplituden der Kontrollqubits beeinflusst.
Die Qubits mit dem Eigenvektor sind für den weiteren Ablauf der Quanten-Phasen-Estimation nicht mehr relevant 
und werden nicht weiter beachtet.
Im Beispiel entspricht die inverse Quanten-Fourier-Transformation einem Hadamard-Gatter da nur ein einzelnes Kontrollqubit existiert:
\[
\frac{1}{\sqrt{2}}
\begin{pmatrix}
  1 \\
  e^{2 i \pi \varphi}
 \end{pmatrix}
 \underrightarrow{H^{\bigotimes 1}}
 \frac{1}{\sqrt{2}}
 \begin{pmatrix}
  1 & 1\\
  1 & -1
 \end{pmatrix}
 \cdot
 \frac{1}{\sqrt{2}}
\begin{pmatrix}
  1 \\
  e^{2 i \pi \varphi}
 \end{pmatrix}
 =
 \frac{1}{2}
 \begin{pmatrix}
  1 + e^{2 i \pi \varphi}\\
  1 - e^{2 i \pi \varphi}
 \end{pmatrix}
\]
Anhand des Ergebnisses des Beispiels ist zu erkennen, dass der Eigenwert praktisch auf die Amplitude vom Kontrollqubit transformiert wird.

Verwendet man ein Phase-Gatter mit \(\varphi  = 0.075\) sieht der Quantenschaltkreis wie in Abbildung~\ref{fig:qpe_1qubit} aus.
Mit dieser Phasenverschiebung sollte man bei einer Messung mit einer Wahrscheinlichkeit von ungefähr
\(0.9455\) den Zustand \(\ket{0}\) erhalten und \(\ket{1}\) mit der Wahrscheinlichkeit \(0.0544\).
Die Ergebnisse von 20.000 Messungen in Abbildung~\ref{fig:qpe_1qubit_Messung} bestätigen die Größenordnung der Wahrscheinlichkeiten.
Jedoch ergeben die Messungen aus Abbildung~\ref{fig:qpe_1qubit_Messung} nicht ganz genau den ausgerechneten Wahrscheinlichkeiten.
Dies liegt an der probabilistischen Natur der Messung.
Bei einer zunehmenden Anzahl an Messungen würden die Ergebnisse an die Wahrscheinlichkeitswerte konvergieren.
Somit benötigt man sehr viele Durchläufe des Quantenalgorithmus um anhand der Messungen ein verlässliches Ergebnis zu erhalten.

Es ist möglich die Präzision der Quanten-Phase-Estimation zu verbessern, 
indem mehr Qubits verwendet werden.
Diese Qubits werden dann als weitere Kontrollqubits verwendet.
Die Anzahl der Qubits für den Eigenvektor bleibt gleich der Bitanzahl, 
die ausreicht, um den Wert des Eigenvektors zu definieren.
Jedes einzelne Kontrollqubit kontrolliert ein \(U^{2^x}\)-Gatter.
Bei \(n\) Kontrollqubits kontrolliert das least-significant-bit ein \(U^{2^0}\)-Gatter,
das darauffolgende ein \(U^{2^1}\)-Gatter,
während das letzte Kontrollqubit ein \(U^{2^{n-1}}\)-Gatter kontrolliert.
Dabei kann \(U^{2^x}\) als \(2^x\) viele \(U\)-Gatter realisiert werden oder 
als ein einzelnes Gatter, welches den Eigenwert \(\lambda\) mit \(2^x\) multipliziert anwendet.
Anschließend wirkt die inverse Quanten-Fourier-Transformation auf alle Kontrollqubits.
Anhand der Messung kann dann \(\varphi\) bestimmt werden.
Der Aufbau der Schaltung ist in Abbildung~\ref{fig:qpe_n_qubit} abgebildet.

Wird die Quanten-Fourier-Transformation wie in Abbildung~\ref{fig:qpe_n_qubit} realisiert,
kann man den Zustand der Kontrollqubits vor der inversen Quanten-Fourier-Transformation wie folgt beschreiben:
\[\frac{1}{\sqrt{N}}[
  (\ket{0} + CU^{2^0}\ket{1}) \bigotimes
  (\ket{0} + CU^{2^1}\ket{1}) \bigotimes 
  ... \bigotimes
  (\ket{0} + CU^{2^{n-1}}\ket{1}) 
]\]
Mit \(U^{n\times n}\ket{x}_n=e^{2\pi i \varphi}\ket{x}_n\) wird der Eigenwert \(e^{2\pi i \varphi}\) wegen des Phase-Kickbacks
über die \(CU\)-Gatter auf die Kontrollqubits übertragen:
\[\frac{1}{\sqrt{N}}[
  (\ket{0} + e^{2\pi i 2^0 \varphi}\ket{1}) \bigotimes
  (\ket{0} + e^{2\pi i 2^1 \varphi}\ket{1}) \bigotimes 
  ... \bigotimes
  (\ket{0} + e^{2\pi i 2^{n-1} \varphi}\ket{1}) 
]\]
Schreibt man \(\varphi\) als Binärbruch:
\[\varphi = \frac{\varphi_n}{2^1} + \frac{\varphi_{n-1}}{2^2} + ... + \frac{\varphi_1}{2^n}\]
Kann die Formel in einer ähnlichen Form wie die Quanten-Fourier-Transformation umgeformt werden~\cite{nielsen_chuang_2010}:
\[\frac{1}{\sqrt{N}}[
  (\ket{0} + e^{2\pi i (\frac{\varphi_n}{2})} ... e^{2\pi i (\frac{\varphi_2}{2^{n-1}})}e^{2\pi i (\frac{\varphi_1}{2^{n}})}\ket{1})  ... 
  (\ket{0} +  e^{2\pi i (\frac{\varphi_2}{2})}e^{2\pi i (\frac{\varphi_1}{4})}\ket{1}) \bigotimes 
  (\ket{0} + e^{2\pi i (\frac{\varphi_1}{2})}\ket{1}) 
]\]
Die Formel besitzt die gespiegelte Struktur wie die Quanten-Fourier-Transformation ohne Swap-Gatter:
\[QFT_{N}\ket{x}_{n} = \frac{1}{\sqrt{N}}(\ket{0} + { e^{2 \pi i \frac{(x_1)}{2}}}\ket{1}) \bigotimes
( \ket{0} + { e^{2 \pi i \frac{(x_2)}{2}\frac{(x_1)}{4}}}\ket{1})...
(\ket{0} + e^{2\pi i (\frac{x_n}{2})} ... e^{2\pi i (\frac{x_2}{2^{n-1}})}e^{2\pi i (\frac{x_1}{2^{n}})}\ket{1})\]
Durch die Verwendung der Swap Gatter kann die Reihenfolge der quanten-Fourier-Transformation gespiegelt werden, 
anschließend sind beide Formeln strukturell identisch.

Wie im Kapitell zur Quanten-Fourier-Transformation erklärt,
transformiert die Quanten-Fourier-Transformation den Zustand der Eingangsqubits \(\ket{x}_n\) in die Phasen der Ausgangsqubits.
Hingegen kehrt die inverse Quanten-Fourier-Transformation diesen Vorgang um, 
indem die Phaseninformationen der Eingangsqubits in Zustände der Standartbasis transformiert werden.

Die Anwendung der inversen Quanten-Fourier-Transformation, inklusive Swap-Gatter, bewirkt also:
\[iQFT(\frac{1}{\sqrt{N}}[
  (\ket{0} + e^{2\pi i (\frac{\varphi_n}{2})} ... e^{2\pi i (\frac{\varphi_2}{2^{n-1}})}e^{2\pi i (\frac{\varphi_1}{2^{n}})}\ket{1})  ... 
  (\ket{0} +  e^{2\pi i (\frac{\varphi_2}{2})}e^{2\pi i (\frac{\varphi_1}{4})}\ket{1}) \bigotimes 
  (\ket{0} + e^{2\pi i (\frac{\varphi_1}{2})}\ket{1}) 
])\]
\[ = \ket{\varphi_1 \varphi_{2}...\varphi_n}_n = \ket{2^n\varphi}_n\]
Schließlich kann \(\varphi\) mit einer division durch \(2^n\) bestimmt werden.

Als Beispiel wird die Quanten-Phase-Estimation für
\(U^{1\times 1}\ket{1}_1=e^{2\pi i \frac{3}{8}}\ket{1}_1\) also mit \(\varphi = \frac{3}{8}\)
beachtet.
Damit der Quantenschaltkreis in Abbildung~\ref{fig:3_qubit_qpe} möglichst gut erkennbar ist,
werden die kontrollierten \(U^{2^x}\)-Gatter werden als Phase-Gatter mit \(P(e^{2^x 2\pi i \frac{3}{8}})\) realisiert.
\begin{figure}
  \caption{3-Kontroll-Qubit QPE}
  \label{fig:3_qubit_qpe}
  \includegraphics[width=\columnwidth]{3_qubit_qpe.png}
  \centering
  \end{figure}
Die Messung in Abbildung~\ref{fig:3_qubit_qpe_measurement} ergibt konsistent bei allen Durchläufen den Zustand \(\ket{3}_3\).
Da \(\ket{3}_3 = \ket{2^3\varphi}_3\) entspricht, kann mit einer Division \(\varphi = \frac{3}{8}\) bestimmt werden.
Es ist zu beachten, dass die Reihenfolge der Wertigkeit der Qubits gleich bleibt.
Normalerweise vertauscht die inverse wie auch die normale Quanten-Fourier-Transformation die Wertigkeiten.
Jedoch wird dieser Effekt mit Swap-Gattern korrigiert.
\begin{figure}
\caption{3-C-Qubit QPE Messergebnis}
\label{fig:3_qubit_qpe_measurement}
\includegraphics[width=\columnwidth]{3_qubit_qpe_measurement.PNG}
\centering
\end{figure}

Im vorherigen Beispiel liefern die Messungen aller Durchläufe den gleichen Zustand mit dem \(\varphi\) eindeutig bestimmbar ist.
Diese Eindeutigkeit tritt auf wenn die verwendete Anzahl an Kontroll-Qubits ausreicht, 
um \(\varphi\) eindeutig zu repräsentieren.
Im oberen Beispiel kann \(\varphi\) eindeutig mit den drei verwendeten Kontrollqubits dargestellt werden:
\({\frac{3}{8} =0 \cdot 2^{-1} + 1\cdot2^{-2} + 1\cdot2^{-3}}\).
Verwendet man nicht ausreichend Kontroll-Qubits kann \(\varphi\) nicht eindeutig repräsentiert werden.
Als Konsequenz wird die Messung ungenau.
Mit hoher Wahrscheinlichkeit kollabieren die Qubits bei einer Messung 
in die darstellbaren Zustände, die den genauen Wert am besten approximieren.
In Abbildung ~\ref*{fig:3_qubit_qpe_measurment_uncertain} sind die Messergebnisse von einer Quanten-Phase-Estimation
abgebildet, die die Phase \(\varphi = \frac{5}{16}\) bestimmen soll.
Da \(\frac{5}{16}\) nicht mit 3-Qubits darstellbar ist, gibt es kein eindeutiges Messergebnis.
Anhand der Messergebnisse ist aber erkennbar, dass die Messungen mit hoher Wahrscheinlichkeit,
zu den bestmöglichen Zuständen kollabieren.
Diese entsprechen \({\frac{2}{8} = \frac{4}{16}}\) und \({\frac{3}{8} = \frac{6}{16}}\), 
also genau die Werte um \(\frac{5}{16}\).


\begin{figure}
  \caption{QPE unpräzises Messergebnis}
  \label{fig:3_qubit_qpe_measurment_uncertain}
  \includegraphics[width=\columnwidth]{3_qubit_qpe_measurment_uncertain.PNG}
  \centering
  \end{figure}












 






 












\section{Shor-Algorithmus}
Im folgenden Kapitel wird der Shor-Algorithmus zum Faktorisieren von Zahlen beschrieben.
Der Inhalt bezieht sich auf den Zweck, die Funktionsweise und den Aufbau des Algorithmus.
Der Aufbau beinhaltet nicht die konkrete Implementierung der Bestandteile des Algorithmus.
Stattdessen werden die Details bezüglich der Implementierung im nächsten Kapitel behandelt.

\subsection{Zweck}
Der Shor-Algorithmus wurde mit dem spezifischen Ziel entwickelt, 
große zusammengesetzte Zahlen effizient auf Quantencomputern in ihre Primfaktoren zu faktorisieren. 
Im Gegensatz zu klassischen Faktorisierungsverfahren, die exponentielle Zeit erfordern~\cite{katz2023}, 
ermöglicht Shor's Ansatz die Faktorisierung in polynomialer Zeit, 
in Bezug auf die Anzahl an Bits der zu faktorisierenden Zahl~\cite{Shor_1997}. 
Dies stellt eine signifikante Beschleunigung gegenüber den besten bekannten klassischen Faktorisierungsverfahren dar.
Der Shor-Algorithmus bekräftigt die These, das Quantencomputer
bestimmte Probleme wesentlich schneller lösen können als ihre klassischen Gegenstücke.

\subsection{Funktionsweise} \label{Funktionsweise}
Der Shor-Algorithmus verwendet zwei Teilberechnungen, die zusammen die Faktorisierung berechnen.
Die erste Teilberechnung erfolgt mittels eines Quantenalgorithmus. 
Der zweite Teil basiert auf einem klassischen Algorithmus.
Im quantenmechanischen Teil des Shor-Algorithmus geht es um die Bestimmung der Ordnung in der multiplikativen Gruppe modulo \(N\).
Hierbei ist anzumerken, dass der Quantenalgorithmus nicht direkt die Primfaktoren der zu faktorisierenden Zahl berechnet.
Stattdessen wird die Eigenschaft ausgenutzt, 
dass das Problem der Faktorisierung äquivalent zu dem Problem der Ordnungsbestimmung ist~\cite{nielsen_chuang_2010}.
Daher impliziert eine effiziente Lösung für die Berechnung der Ordnung eine ebenso effiziente Methode zur Faktorisierung.
Der nachfolgende klassische Algorithmus verwendet die ermittelte Ordnung, 
um die Primfaktoren abzuleiten. 
Beide Teilberechnungen, sowohl die quantenmechanische als auch die klassische, 
führen ihre Berechnungen in polynomialer Zeit durch. 
Daher liegt die gesamte Laufzeit des Shor-Algorithmus ebenfalls in einer polynomialen Größenordnung.

\subsection{Ordnungsbestimmung}
Zu bestimmen sind die Primfaktoren der Zahl \(N\).
Zuerst wird ein \(a\) mit \(0 < a < N\) gewählt.
Falls der ungewöhnlichen Fall eintritt, dass \(a\) nicht teilerfremd zu \(N\) ist, entspricht \(a\) einem der Primfaktoren.
Anschließend wird die Ordnung beziehungsweise Periode \(p\) der Funktion \({f(x) = a^x \mod N}\) mit einem Quantenalgorithmus bestimmt:

Die Periode \(p\) beschreibt das kleinste ganzzahlige Element mit \({p > 0}\), für das gilt: \({f(p) = 1 \mod N}\).
\[
\begin{tabular}{l|llllll}
    x     &     0     &     1       &     2      &      3   &  4 &  5  \\ \hline
    \(7^x \mod 15\)    &      1     &        7     &       4     &     13     &  1 &  7 
\end{tabular} \longmapsto p = 4
\]

Im Wesentlichen handelt es sich bei der quantenmechanischen Berechnung des Shor-Algorithmus um die Quanten-Phase-Estimation.
Die Architektur dieser spezifischen Quanten-Phase-Estimation korrespondiert weitgehend mit der in Abschnitt~\ref{Quanten-Phase-Estimation} vorgestellten Struktur.
Hierbei ersetzen speziell für die gegebene Anwendung definierte \(U\)-Gatter die allgemeinen.

Für den konkreten Kontext der Periodenberechnung, realisieren die \(U\)-Gatter die Transformation:
\[U\ket{y} = \ket{ay \mod N}\] 
Die Ausführung der Quanten-Phase-Estimation erfordert die Erzeugung eines Eigenvektors der Transformation \(U\).
Da die Quanten-Phase-Estimation \(\varphi\) aus dem Eigenwert extrahiert, 
darf der Eigenvektor nicht den trivialen Eigenwert 1 besitzen.
Stattdessen ist es notwendig, dass die Periode der Transformation im Eigenwert enthalten ist.

Wie in~\cite*{nielsen_chuang_2010} gezeigt wird, gibt es zu \(U\) Eigenvektoren \(\ket{u_s}\), 
mit \(0 \leq s \leq p-1\): 
\[\ket{u_s} \equiv
\frac{1}{\sqrt{p}}
\sum_{k=0}^{p-1} e^{\frac{-2 \pi i s k}{p}\ket{a^k \mod N}} %ich soll die Formel mit der Summe sein
\]
Für \(a=7\) bei \(N=15\) entspricht \(r=4\).
Ein Eigenvektor zu \(s=1\) lautet dann:
\[
    \ket{u_1}_4 =
    \frac{1}{\sqrt{4}}(
        \ket{1}_4 + 
        e^{-\frac{2 \pi i}{4}}\ket{7}_4 + 
        e^{-\frac{4 \pi i}{4}}\ket{4}_4+ 
        e^{-\frac{6 \pi i}{4}}\ket{13}_4
    )
    \]
\[
    U\ket{u_1}_4 =
    \frac{1}{\sqrt{4}}(
        \ket{7}_4 + 
        e^{-\frac{2 \pi i}{4}}\ket{4}_4 + 
        e^{-\frac{4 \pi i}{4}}\ket{13}_4+ 
        e^{-\frac{6 \pi i}{4}}\ket{1}_4
    )
    \]
\[
    U\ket{u_1}_4 =
    e^{\frac{2 \pi i}{4}}
    \cdot
    \frac{1}{\sqrt{4}}(
        e^{-\frac{2 \pi i}{4}}\ket{7}_4 + 
        e^{-\frac{4 \pi i}{4}}\ket{4}_4 + 
        e^{-\frac{6 \pi i}{4}}\ket{13}_4+ 
        e^{-\frac{8 \pi i}{4}}\ket{1}_4
    )
    \]
\[
    U\ket{u_1}_4 =
    e^{\frac{2 \pi i}{4}}
    \cdot
    \frac{1}{\sqrt{4}}(
        e^{-\frac{8 \pi i}{4}}\ket{1}_4
        e^{-\frac{2 \pi i}{4}}\ket{7}_4 + 
        e^{-\frac{4 \pi i}{4}}\ket{4}_4 + 
        e^{-\frac{6 \pi i}{4}}\ket{13}_4 )
    =
    e^{\frac{2 \pi i}{4}} \cdot
    \ket{u_1}_4
    \]
Das kann verallgemeinert werden:
\[U\ket{u_s} = e^{\frac{2 \pi i s}{p}}\ket{u_s}\]

Wie man in der Definition von \(\ket{u_s}\) %ref zu Formel mit der Summe
sieht, 
benötigt die Initialisierung eines Eigenvektors \(\ket{u_s}\) mit einem konkreten \(s\) die Periode \(p\).
Man kann diese Problematik jedoch umgehen indem man anstelle eines einzelnen Eigenvektors \(\ket{u_s}\)
eine Superposition verwendet, die alle \(\ket{u_s}\) umfasst.
Die Superposition entspricht~\cite*{nielsen_chuang_2010}:
\[\frac{1}{\sqrt{p}} \sum_{s=0}^{r-1}\ket{u_s} = \ket{1}\] 
Also werden die Qubits, die für den Eigenvektor bestimmt sind, mit dem Zustand \(\ket{1}\) initialisiert.

Die Superposition der Eigenvektoren hat zur Folge, 
dass nach der inversen Quanten-Fourier-Transformation, also am Ende der Quanten-Phase-Estimation,
ebenfalls eine Superposition mit dem \(\varphi_s\) der Eigenwerte \(\lambda_{u_s}\) aller möglichen Eigenvektoren \(\ket{u_s}\) existiert.
Sei \(k\) die Anzahl an Kontroll-Qubits dann:
\[
    \frac{1}{\sqrt{p}} \sum_{s=0}^{p-1}\ket{2^k \cdot \frac{s}{p}}  = 
    \frac{1}{\sqrt{p}} (\ket{0} + \ket{2^k \cdot \frac{1}{p}} + \ket{2^k \cdot \frac{2}{p}} ... + \ket{2^k \cdot \frac{p-1}{p}})
    \]


Bei einer Messung wird also zufällig eines der \(\varphi_s \approx \frac{s}{p}\) gemessen.

Die Genauigkeit von \(\frac{s}{p}\) gegenüber dem \(\varphi_s\) hängt davon ab,
mit wie viele \(U^{2^x}\)-Gatter beziehungsweise mit wie viele Kontroll-Qubits,
die Quanten-Phase-Estimation ausgeführt wurde.

Mit einer ausreichenden Anzahl an verwendeten Kontroll-Qubits, 
können die Zustände vollkommen beschrieben werden und 
somit bei einer Messung die genauen \(\varphi_s\) gefunden.  

Falls nicht ausreichend Kontroll-Qubits vorhanden sind,
wird die Messung mit hoher Wahrscheinlichkeit, 
in den Zustand kollabieren der dem genauen Ergebnis am nächsten ist. 
Die Ergebnisse sind dann jedoch probabilistischer Natur, 
vergleichbar mit dem Beispiel aus dem Quanten-Phase-Estimation Kapitel, 
in Abbildung~\ref*{fig:3_qubit_qpe_measurment_uncertain}.

Anhand des gemessenen \(\frac{s}{p}\) erfolgt die Primfaktorzerlegung in einer klassischen Nachberechnung.

\subsection{Klassische Nachberechnung}
%Detailierter bezüglich Wahrscheinlichkeiten die Periode bzw Faktoren zu finden
Wie der Name dieses Abschnittes vermuten lässt, 
wird die Nachberechnung in der Regel mit einem klassischen Algorithmus durchgeführt.

Aus dem Messergebnis des Quanten-Phase-Estimation soll die Phase extrahiert werden.
Anhand der Messung sind die ersten \(k\) Bits von \(\varphi_s\) bekannt.
Dabei steht \(k\) für die Genauigkeit der Quanten-Phase-Estimation.
Die Genauigkeit \(k\) wird durch die Anzahl an Kontroll-Qubits festgelegt.
Sollten \(k\) Bits nicht vollständig ausreichen um \(\varphi_s\)  zu beschreiben, 
wird man bei der Messung eine Kommazahl erhalten die nah von \(\varphi_s\) liegt,
dieser aber nicht ganz entspricht.
Wendet man den Kettenbruch-Algorithmus auf das Messergebnis an,
wird von der Kommazahl aus, der nächste ganzzahlige Bruch gefunden.
Man kann zeigen, dass die Verwendung von \(k = 2n+1\) Kontrollqubits eine ausreichende Genauigkeit liefert,
so dass das Messergebnis unter Verwendung des Kettenbruch-Algorithmus zu einem Näherungsbruch \(\frac{s}{p}\) von \(\varphi_s\) führt~\cite*{nielsen_chuang_2010}.

Nichtsdestotrotz kann selbst bei der Verwendung von ausreichend Kontroll-Qubits ein \(\frac{s'}{p'}\) berechnet werden,
welches nicht die Periode \(p\) enthält.
Dies tritt auf wenn \(s\) und \(p\) einen gemeinsamen Teiler haben der die Kürzung von \(\frac{s}{p}\) auf \(\frac{s'}{p'}\) ermöglicht.

Findet man in der ersten Messung \(\frac{s'}{p'}\) und in einer zweiten \(\frac{s''}{p''}\), 
wobei \(s'\) und \(s''\) keine Faktoren teilen,
so kann aus dem kleinsten gemeinsamen Vielfachen von \(p'\) und \(p''\) \(p\) berechnen.

Man kann zeigen, 
dass bei \(2\log(N)\) Messungen, die Wahrscheinlichkeit sehr hoch ist, mindestens einmal ein \(\frac{s}{p}\)
zu messen, bei dem \(s\) und \(p\) teilerfremd sind~\cite*{nielsen_chuang_2010}.

Ob die korrekte Periode gefunden wurde, kann mit \(a^p = 1 \mod N\) geprüft werden.

Nach einige fehlgeschlagenen Versuchen(z.B. \(2\log(N)\)) wiederholt man die Suche mit einem anderen \(a\).

Sobald die korrekte Periode \(p\) gefunden wurde, 
können die Primfaktoren von \(N\)mit dem gemeinsamen Teiler(gcd) berechnet werden:
\[gcd(a^{\frac{p}{2}}-1, N), gcd(a^{\frac{p}{2}}+1, N)\]
Dies schlägt nur fehl falls \(r\) ungerade ist
oder falls \(a^{\frac{p}{2}} = -1 \mod N\) erfüllt~\cite*{Shor_1997}.
Die Wahrscheinlichkeit dass einer der beiden genannten Fälle eintritt beträgt \(1-\frac{1}{2^k}\), 
wobei \(k\) die Anzahl an unterschiedlicher Primfaktoren von \(N\) angibt~\cite*{Shor_1997}.
In einem solchen Fall, wiederholt man die Periodenberechnung mit einem anderen \(a\).





\section{Implementierung}

\subsection{Quantenalgorithmus}
Wie im Kapitel~\ref*{Funktionsweise} zur Funktionsweise erklärt,
wird für die Periodenbestimmung die Quanten-Phase-Estimation genutzt.

Um den Quanten-Phase-Estimation Algorithmus für die Periodenberechnung zu nutzen,
benötigt man ein Gatter \(U\) welches die modulare Multiplikation \(U\ket{y} = \ket{ay \mod N}\), 
als eine unitär Transformation realisiert.
Mit den passenden \(U\)-Gatter wird der Quantenschaltkreis wie in Abbildung~\ref{fig:shor_n_qubit} strukturiert.
\begin{figure}
    \caption{QPE für Shor~\cite{anonymousket}}
    \label{fig:shor_n_qubit}
    \includegraphics[width=\columnwidth]{shor_n_qubit.png}
    \centering
    \end{figure}

Die Realisierung der Transformation bedingt die Implementierung einiger arithmetischer Operationen in Form eines Quantenschaltkreises. 
Diese fungieren als Bausteine, die zusammengesetzt zur Konstruktion des übergeordneten Quantenschaltkreises für die modulare Multiplikation beitragen. 
Zu den erforderlichen arithmetischen Operationen gehört die Addition, Subtraktion sowie die modulare Addition.

In den folgenden Abschnitten werden die untergeordneten arithmetischen Operationen bis hin zur modularen Multiplikation implementiert.

\subsubsection{Addition}
Der Quantenschaltkreis für die Addition bildet das Fundament der \(U\)-Gatter und 
stellt einen der am häufigsten verwendeten Bausteine dar. 
Deswegen hat die Implementierung der Addition einen erheblichen Einfluss auf den Ressourcenbedarf des gesamten Quantenalgorithmus
und sollte daher möglichst effizient implementiert werden.

Eine Möglichkeit, die Addition als Quantenschaltkreis zu realisieren, 
besteht im Nachbau eines klassischen Schaltkreises aus Volladdierern. 
Da es nicht möglich ist, 
die notwendige klassischen Gatter wie AND und OR als unitäre Transformation mit nur zwei Qubits darzustellen~\cite{Hoever2023QC},
werden zusätzliche Hilfsqubits benötigt.
Die zusätzlichen Hilfsqubits bewirken, dass der Nachbau eines klassischen Schaltkreis für die Addition zweier \(n\)-Bit Zahlen, 
also solche der Größenordnung \(2^n\), mindestens \(3n\) Qubits benötigt~\cite{zalka1998fast}.

Eine effizientere Methode, die ohne Hilfsqubits auskommt, ist die Quanten-Addition~\cite{draper2000addition}. 
Die Quanten-Addition führt die Berechnung auf quantenmechanische Weise durch. 
Im Wesentlichen wird dabei die Addition in der Fourier-Basis berechnet, 
wobei die Phasen der Qubits eines Summanden mit kontrollierte Phasenverschiebungen auf die Qubits des anderen Summanden wirkt.

Im Folgenden wird ein Beispiel für die Quanten-Addition zweier Qubit-Register \(\ket{a}_3\) und \(\ket{b}_3\), 
jeweils bestehend aus drei Qubits, betrachtet:
\begin{figure}[H]
    \caption{Quantum-Addition}
    \label{fig:3_qubit_quantum_add}
    \includegraphics[width=\columnwidth]{3_qubit_quantum_add.png}
    \centering
    \end{figure}
Die Registermarkierungen in der Mitte von Abbildung~\ref*{fig:3_qubit_quantum_add} unterteilen die Darstellung in zwei Hälften.
Die linke Hälfte repräsentiert die Quanten-Fourier-Transformation, 
während die rechte Hälfte die Quanten-Addition zeigt.

Wie man an der Struktur der Quanten-Addition erkennen kann,
ist die Anordnung der Gatter fast identisch mit der Quanten-Fourier-Transformation.
Ein Unterschied besteht darin, 
dass die Hadamard-Gatter durch kontrollierte \(P(\pi)\) Phasen-Gatter ersetzt wurden.
Sowohl das Hadamard-Gatter als auch das \(P(\pi)\) Phasen-Gatter erzeugen eine relative Phase von \(e^{\pi i}\).
Ein weiterer Unterschied zur Quanten-Fourier-Transformation besteht darin, 
dass die Phasen-Gatter nicht durch das gleiche Register\(\ket{a}_3\) kontrolliert werden, 
auf das die Gatter auch wirken.
Stattdessen kontrollieren die Qubits des Registers \(\ket{b}_3\) die Phasen-Gatter.
Dabei wird das \(P(\pi)\) Phasen-Gatter durch das Qubit des \(\ket{b}_3\) kontrolliert,
welches die selbe Wertigkeit hat wie das Zielqubit des \(\ket{a}_3\) Registers.
Jedes weitere kontrollierte Phasen-Gatter für das gleiche Zielqubit 
wird fortlaufend von dem nächstkleineren Qubit von \(\ket{b}_3\) kontrolliert.

Im Prinzip handelt es sich bei dieser Quantenschaltung um eine Anwendung derselben Phasenverschiebungen 
wie bei der Quanten-Fourier-Transformation. 
Der grundlegende Unterschied liegt darin, 
dass diese Phasenverschiebungen kontrolliert auf ein anderes Quantenregister angewendet werden.

Die Wirkung der Quanten-Addition wird anhand der Abbildung~\ref*{fig:3_qubit_quantum_add} verdeutlicht:
Am Anfang der linken Hälfte befinden sich beide Register in der Standardbasis.
Auf das Zielregister \(\ket{a}_3\) wirkt die Quanten-Fourier-Transformation ohne Swap Gatter.
Dadurch befindet sich \(\ket{a}_3\) nun in der Fourier-Basis \(\Phi\), also \(\ket{\Phi(a)}_3\):
\[\ket{\Phi(a)}_3 = \frac{1}{\sqrt{8}} [ (\ket{0} + { e^{\frac{2 \pi i (2^0a_1)}{2^1}}}\ket{1} ) \bigotimes
( \ket{0} + { e^{\frac{2 \pi i (2^1a_2+2^0a_1)}{2^2}}}\ket{1} ) \bigotimes
( \ket{0} + { e^{\frac{2 \pi i (2^{2}a_3 +2^1a_2+2^0a_1)}{2^3}}}\ket{1} ) ] \]
Anschließend wirkt auf das hinterste Tensorprodukt ein \(P(\pi)\) Phasen-Gatter,
welches durch \(\ket{b_3}_1\) kontrolliert wird.
Wenn sich \(\ket{b_3}_1\) im Zustand \(\ket{0}_1\) befindet, passiert nichts.
Wenn es sich im Zustand \(\ket{1}_1\) befindet, dass das Phasen-Gatter angewendet wird.
Dieses Verhalten kann man für beide Fälle mit den entsprechenden Matrizen 
\(\begin{pmatrix}
    1 & 0 \\
    0 & e^{\pi i b_3}
  \end{pmatrix}\)
  beziehungsweise  
  \(\begin{pmatrix}
    1 & 0 \\
    0 & e^{\frac{2\pi i (2^2b_3)}{2^3}}
  \end{pmatrix}\)
beschreiben.
Schreibt man das hinterste Tensorprodukt als Vektor, ergibt sich die folgende Formulierung:
\[\frac{1}{\sqrt{2}}( \ket{0} + { e^{\frac{2 \pi i (2^{2}a_3 +2^1a_2+2^0a_1)}{2^3}}}\ket{1}) \equiv
\frac{1}{\sqrt{2}}
\begin{pmatrix}
     1  \\
     e^{\frac{2 \pi i (2^{2}a_3 +2^1a_2+2^0a_1)}{2^3}}
  \end{pmatrix}
    \]
Dann wird durch das Ergebnis der Verrechnung mit dem Phasen-Gatter deutlich, 
dass die Addition im Wesentlichen in der Phase des Quantenzustands stattfindet:
\[\begin{pmatrix}
    1 & 0 \\
    0 & e^{\frac{2\pi i (2^2b_3)}{2^3}}
  \end{pmatrix}
    \cdot
\frac{1}{\sqrt{2}}
\begin{pmatrix}
    1  \\
     e^{\frac{2 \pi i (2^{2}a_3 +2^1a_2+2^0a_1)}{2^3}}
  \end{pmatrix}
  =
  \frac{1}{\sqrt{2}}
  \begin{pmatrix}
    1  \\
     e^{\frac{2 \pi i (2^{2}(a_3+b_3) +2^1a_2+2^0a_1)}{2^3}}
  \end{pmatrix}
\]
Wie in der Abbildung~\ref*{fig:3_qubit_quantum_add} erkenntlich,
wirken auf das hinterste Tensorprodukt auch noch die beiden Phasen-Gatter \(P(\frac{\pi}{2})\) und \(P(\frac{\pi}{4})\) mit:
\[
    P(\frac{\pi}{2}) = 
\begin{pmatrix}
    1 & 0 \\
    0 & e^{\frac{\pi}{2} i b_2}
  \end{pmatrix}
  =
  \begin{pmatrix}
    1 & 0 \\
    0 & e^{\frac{2\pi i (2^1b_2)}{2^3}}
  \end{pmatrix}
  ~;~
P(\frac{\pi}{4}) = 
\begin{pmatrix}
    1 & 0 \\
    0 & e^{\frac{\pi}{4} i b_1}
  \end{pmatrix}
  =
  \begin{pmatrix}
    1 & 0 \\
    0 & e^{\frac{2\pi i (2^0b_1)}{2^3}}
  \end{pmatrix}
\]
\[
    \begin{pmatrix}
        1 & 0 \\
        0 & e^{\frac{2\pi i (2^0b_1)}{2^3}}
      \end{pmatrix}
      \cdot
      \begin{pmatrix}
        1 & 0 \\
        0 & e^{\frac{2\pi i (2^1b_2)}{2^3}}
      \end{pmatrix}
      \cdot
      \frac{1}{\sqrt{2}}
      \begin{pmatrix}
        1  \\
         e^{\frac{2 \pi i (2^{2}(a_3+b_3) +2^1a_2+2^0a_1)}{2^3}}
      \end{pmatrix}
      =
      \frac{1}{\sqrt{2}}
      \begin{pmatrix}
        1  \\
         e^{\frac{2 \pi i (2^{2}(a_3+b_3) +2^1(a_2+b_2)+2^0(a_1+b_1))}{2^3}}
      \end{pmatrix}
\]
Wendet man alle weiteren Phasen-Gatter auf das vollstände Tensorprodukt an, 
erhält man:
\[
    \frac{1}{\sqrt{8}} [ (\ket{0} + { e^{\frac{2 \pi i (2^0(a_1+b_1))}{2^1}}}\ket{1} ) \bigotimes
( \ket{0} + { e^{\frac{2 \pi i (2^1(a_2+b_2)+2^0(a_1+b_1))}{2^2}}}\ket{1} ) \bigotimes
( \ket{0} + { e^{\frac{2 \pi i (2^{2}(a_3+b_3) +2^1(a_2+b_2)+2^0(a_1+b_1))}{2^3}}}\ket{1} ) ]
\]
Setzt man in diese Formel zwei Zahlen in Binärschreibweise ein, 
wird man den selben Zustand erhalten,
wie wenn man die Summe der beiden Zahlen in die Formel der Quanten-Fourier-Transformation einsetzt.
Beispielsweise sei \(a = 3\) also binär \(a_3 = 0\), \(a_2 = 1\),\(a_1 = 1\) und 
\(b = 1\) also \(b_3 = 0\), \(b_2 = 0\),\(b_1 = 1\):
\[
\frac{1}{\sqrt{8}} [ (\ket{0} + { e^{\frac{2 \pi i (2^0(1+1))}{2^1}}}\ket{1} ) \bigotimes
( \ket{0} + { e^{\frac{2 \pi i (2^1(1+0)+2^0(1+1))}{2^2}}}\ket{1} ) \bigotimes
( \ket{0} + { e^{\frac{2 \pi i (2^{2}(0+0) +2^1(1+0)+2^0(1+1))}{2^3}}}\ket{1} ) ]
\]
\[
=\frac{1}{\sqrt{8}} [ (\ket{0} + \ket{1} ) \bigotimes
( \ket{0} +   \ket{1} ) \bigotimes
( \ket{0} +  e^{\pi i }\ket{1} ) ]
\]
Das dies tatsächlich die Summe in Fourier-Basis entspricht, 
wird deutlich wenn man das selbe Tensorprodukt aus der Quanten-Fourier-Transformation bildet:
\[
    QFT(\ket{c}_3)
    \frac{1}{\sqrt{8}} [ (\ket{0} + { e^{\frac{2 \pi i (2^0(c_1))}{2^1}}}\ket{1} ) \bigotimes
( \ket{0} + { e^{\frac{2 \pi i (2^1(c_2)+2^0(c_1))}{2^2}}}\ket{1} ) \bigotimes
( \ket{0} + { e^{\frac{2 \pi i (2^{2}(c_3) +2^1(c_2)+2^0(c_1))}{2^3}}}\ket{1} ) ]
\]
Die Summe von \(a\) und \(b\) entspricht \(c = 4\) also \(c_3 = 1,~c_2 = 0,~c_1=0\):
\[
    QFT(\ket{4}_3)
    \frac{1}{\sqrt{8}} [ (\ket{0} + { e^{\frac{2 \pi i (2^0(0))}{2^1}}}\ket{1} ) \bigotimes
( \ket{0} + { e^{\frac{2 \pi i (2^1(0)+2^0(0))}{2^2}}}\ket{1} ) \bigotimes
( \ket{0} + { e^{\frac{2 \pi i (2^{2}(1) +2^1(0)+2^0(0))}{2^3}}}\ket{1} ) ]
\]
\[
    = 
    \frac{1}{\sqrt{8}} [ (\ket{0} + { e^{\frac{2 \pi i (0)}{2^1}}}\ket{1} ) \bigotimes
( \ket{0} + { e^{\frac{2 \pi i (0)}{2^2}}}\ket{1} ) \bigotimes
( \ket{0} + { e^{\frac{2 \pi i (2^{2}(1))}{2^3}}}\ket{1} ) ]
\]
\[
=\frac{1}{\sqrt{8}} [ (\ket{0} + \ket{1} ) \bigotimes
( \ket{0} +   \ket{1} ) \bigotimes
( \ket{0} +  e^{\pi i }\ket{1} ) ]
\]
Das Ergebnis der Quanten-Addition zweier Summanden \(a=3,~b=1\), 
jeweils in einem Register mit drei Qubits,
ist somit identisch mit dem Zustand, 
der durch die Anwendung der Quanten-Fourier-Transformation auf ein Register 
aus ebenfalls drei Qubits mit der Summe der beiden Zahlen entsteht.

Mit einer anschließenden inversen Quanten-Fourier-Transformation, 
kann die Summe in die Standardbasis und somit in einen Messbaren Zustand transformiert werden:
\[
iQFT(\ket{\Phi(4)_3})
\equiv
 iQFT(\frac{1}{\sqrt{8}} [ (\ket{0} + \ket{1} ) \bigotimes
( \ket{0} +   \ket{1} ) \bigotimes
( \ket{0} +  e^{\pi i }\ket{1} ) ]) 
=
\ket{4}_3
\]

Das Zielregister sollte aus genügend Qubits bestehen, 
damit die Summe vollständig erfasst werden kann.
Andernfalls kommt es zum overflow mit \(a + b \mod 2^n\), 
wobei \(n\) die Anzahl an Qubits des Zielregisters beschreibt~\cite{beauregard2003circuit}.

Für die Realisierung der modularen Multiplikation wird zu keinem Zeitpunkt der Berechnung eine Addition zweier Zwischenergebnisse benötigt. 
Genauer gesagt, ist es nicht nötig, ein Quantenregister auf ein anderes zu addieren. 
Stattdessen wird die Quanten-Addition benutzt, um eine vorab bekannte Zahl auf ein Quantenregister zu addieren.

Bei der Quanten-Addition mit zwei Register wie in Abbildung~\ref{fig:3_qubit_quantum_add} erfolgt die Phasenverschiebung kontrolliert,  
also in Abhängigkeit des Inhaltes von Register \(\ket{b}_3\).
Ist der Inhalt von Register \(\ket{b}_3\) vorab bekannt,
können gewöhnliche Phasen-Gatter anstelle von kontrollierten verwendet werden~\cite{beauregard2003circuit}.

Wenn bei der Quanten-Addition mit zwei Registern ein Phasen-Gatter aufgrund des zugehörigen Kontrollqubits im Zustand \(\ket{1}\) angewendet wird,
wird es in der Variante mit einem einzelnen Register als gewöhnliches Phasen-Gatter verwendet.
Ist das Kontrollqubit hingegen im Zustand \(\ket{0}\), 
wodurch das Phasen-Gatter bei der Quanten-Addition mit zwei Registern nicht zur Anwendung kommt, 
wird dieses Phasen-Gatter in der Variante mit nur einem Register weggelassen.

\begin{figure}[H]
    \caption{Quantum-Addition fixierte Phasenverschiebungen}
    \label{fig:3_qubit_fixed_quantum_addition}
    \includegraphics[width=\columnwidth]{3_qubit_fixed_quantum_addition.png}
    \centering
    \end{figure}
In Abbildung~\ref{fig:3_qubit_fixed_quantum_addition} ist die Quanten-Addition für ein 3-Qubit Register abgebildet.
Die blauen Phasen-Gatter sorgen für die Quanten-Addition mit einem fixierten Wert von \(3\).
Vergleicht man die Abbildung~\ref{fig:3_qubit_fixed_quantum_addition} mit der Abbildung~\ref{fig:3_qubit_quantum_add} fällt auf, 
dass das aller erste Phasen-Gatter der Quanten-Addition nicht vorkommt.
Im Quantenschaltkreis der Abbildung~\ref{fig:3_qubit_quantum_add} würde ein Registerinhalt von \(b = 3\) das Kontrollqubit \(b_3\) nicht setzen. 
Somit kommt das erste Phasen-Gatter der Quanten-Addition nicht zum Einsatz und 
wird deswegen in der Variante wie in Abbildung~\ref{fig:3_qubit_fixed_quantum_addition} weggelassen. 

Des weiteren ist es möglich, 
den Quantenschaltkreis der Quanten-Addition für eine vorab festgelegte Zahl ressourcensparender zu realisieren.
Die Optimierung bezieht sich dabei auf die Anzahl der verwendeten Gatter.
Wird die Quanten-Addition für eine feste Zahl realisiert, 
können für jedes einzelne Qubit die angewendeten Phasenverschiebungen vorab zusammengerechnet werden.
Demnach bietet es sich an, 
die zusammengerechneten Phasenverschiebung in ein einzelnes Phasen-Gatter zusammen zu fassen.
Nach diesem Prinzip, benötigt man für eine Quanten-Addition maximal ein einzelnes Phasen-Gatter pro Qubit.

\begin{figure}[H]
  \caption{Ressourcensparende Quantum-Addition}
  \label{fig:3_qubit_fixed_quantum_addition_opt}
  \includegraphics[width=\columnwidth]{3_qubit_fixed_quantum_addition_opt.png}
  \centering
  \end{figure}
Der Quantenschaltkreis in Abbildung~\ref{fig:3_qubit_fixed_quantum_addition_opt} führt die identische 
Berechnung durch, wie der Quantenschaltkreis aus Abbildung~\ref{fig:3_qubit_fixed_quantum_addition}.
Darüber hinaus, werden die benötigten Phasenverschiebungen mit einem einzigen Phasen-Gatter pro Qubit realisiert.

Die Implementierung nutzt die ressourceneffiziente Variante der Quanten-Addition.
In der Funktion \texttt{A\_Gate} wird ein Gatter erzeugt, 
das sämtliche für die Quanten-Addition erforderlichen Phasen-Gatter umfasst.
Der Code zur Funktion ist in Abbildung~\ref{code:QuantumAdd} dargestellt.
\begin{figure}[H]
  \caption{Quantum-Addition in Qiskit}
  \label{code:QuantumAdd}
\begin{minted}[linenos]{python}    
def A_Gate(a_bin: list[int]) -> qiskit.circuit.gate:
    A_Gate = qiskit.QuantumCircuit(len(a_bin))
    theta_list = [0.0]*len(a_bin)
    for target_bit in range(len(a_bin)):
        exponent = 1
        for control_bit in reversed(range(target_bit+1)):
            if a_bin[control_bit] == 1:
                theta_list[target_bit]+= 2*pi/(2**(exponent))
            exponent+=1
    for qubit_index in range(len(a_bin)):
        A_Gate.append(P_Gate(theta_list[qubit_index]),[qubit_index])
    A_Gate = A_Gate.to_gate()
    A_Gate.name = " Add(" + str (binToDez(a_bin) )+ ")"
    return A_Gate 
  \end{minted}
\end{figure}
Der Funktion \texttt{A\_Gate} wird der zu addierende Summand im Binärformat übergeben, 
wobei am Index Null das Least-Significant-Bit liegt.
Abhängig von der Anzahl der Bits des Summanden wird ein Quantenschaltkreis mit der gleichen Anzahl an Qubits erzeugt. 
In den Zeilen 4 bis 9 werden die Phasenverschiebungen, die auf ein einzelnes Qubit wirken sollen, berechnet und akkumuliert.
Anschließend wird in den Zeilen 10 bis 11 auf jedes Qubit des Quantenschaltkreises ein individuelles Phasen-Gatter angewendet.
Die Phasenverschiebung eines einzelnen Phasen-Gatters ergibt sich aus der vorherigen Akkumulation in den Zeilen 4 bis 9.
Abschließend wird der Quantenschaltkreis in ein Gatter umgewandelt und mit einer passenden Bezeichnung versehen.

\subsubsection{Subtraktion}
Die Quanten-Addition besteht ausschließlich aus unitären Gattern 
und ist daher selbst auch unitär. 
Diese Eigenschaft vereinfacht die Implementierung der Subtraktion. 
Indem die Quanten-Addition invertiert wird, ergibt sich ein Quantenschaltkreis, 
der eine Subtraktion in der Fourier-Basis ausführt. 
Damit hat man praktisch einen Quantenschaltkreis der Quanten-Subtraktion.

Genau wie bei der Quanten-Addition kann es auch bei der Quanten-Subtraktion zu einem Overflow,
beziehungsweise im konkreten Kontext, zu einem Underflow kommen.
Bei der Quanten-Subtraktion tritt dieser Effekt ein, 
falls der Minuend im Zielregister kleiner als der Subtrahend ist.

Dieser Effekt kann verwendet werden, um herauszufinden, 
ob der Subtrahend größer ist als der Minuend~\cite{beauregard2003circuit}.
Angenommen man hat zwei Zahlen die maximal der Größenordnung \(2^n\) entsprechen und 
ein Zielregister welches aus \(n+1\) Qubits besteht.
Der Subtrahend \(b\) befindet sich im Zielregister also \(\ket{\phi(b)}_{n+1}\), 
worauf die Quanten-Subtraktion mit dem Minuend \(a\) wirkt.
Dann existieren zwei mögliche Fälle:
\[b \geq a~\rightarrow~\ket{\phi(b-a)}_{n+1};~
b < a~\rightarrow~\ket{\phi(2^{n+1}-(a-b))}_{n+1}
  \]
Das Most-Significant Bit ist ausschließlich im zweiten Fall gesetzt und kann somit, 
nachdem das Register in die Standardbasis transformiert wird, 
als eine Art Borrow-Bit entspricht.

Die Implementierung der Quanten-Subtraktion ist aufgrund der \texttt{inverse} Funktion von Qiskit unkompliziert.
Wendet man die \texttt{inverse} Funktion auf ein Gatter der Quanten-Addition der \texttt{A\_Gate} Funktion an, 
wird dieses in die Quanten-Subtraktion invertiert.
Der Code davon ist in~\ref{code:QuantumSub} abgebildet.
\begin{figure}[H]
  \caption{Quantum-Subtraktion in Qiskit}
  \label{code:QuantumSub}
\begin{minted}[linenos]{python}    
def S_Gate(subtrahend_bin: list[int]) -> qiskit.circuit.gate:
    S_Gate = A_Gate(subtrahend_bin).inverse()
    S_Gate.name = "  Sub(" + str (binToDez(subtrahend_bin) )+ ")"
    return S_Gate
  \end{minted}
\end{figure}

\subsubsection{Modulare Addition}
Die Gatter der Quanten-Addition und der Quanten-Subtraktion ermöglichen die Konstruktion einer unitären Transformation, 
die die Berechnung der modularen Addition realisiert.
Gemeinsam bilden sie einen Quantenschaltkreis, 
der das Ergebnis der Berechnung \(\ket{\Phi(a+b)\mod N}\) für \(a, b < N\) im Zielregister speichert.

\begin{figure}[H]
  \caption{Modulare Addition nach Beauregard~\cite{beauregard2003circuit}}
  \label{fig:modulare_addition_paper}
  \includegraphics[width=\columnwidth]{Modular_adder_Paper.PNG}
  \centering
  \end{figure}
Abbildung~\ref{fig:modulare_addition_paper} zeigt das verwendete Konzept, 
das für die Implementierung der modularen Addition als Bauplan diente.
In der Grafik sind zwei Arten von \(\phi ADD\)-Gattern zu sehen.
Dabei handelt es sich um die Quanten-Addition, erkennbar mit dem fettgedruckten Strich auf der rechten Seite und 
der Quanten-Subtraktion mit einem fettgedruckten Strich auf der linken Seite.
Sowohl die Quanten-Addition als auch die Quanten-Subtraktion addieren beziehungsweise subtrahieren eine fixe Zahl und 
sind somit die zuvor vorgestellten Varianten mit gewöhnlichen Phasen-Gattern.
Der Quantenschaltkreis zur modularen Addition agiert als Baustein in einem größeren Gesamtbild, 
weshalb die zwei Kontroll-Qubits \(\ket{c_1}\) und \(\ket{c_2}\) eingebaut werden, 
die im weiteren Verlauf Anwendung finden.
Das Eingaberegister stellt auch das Zielregister dar und 
ist initial bereits in der Fourier-Basis mit \(\ket{\phi(b)}\).
Das Zielregister wird um ein weiteres Qubit erweitert, 
welches das Most-Significant-Bit des Zielregisters dargestellt.
Dieses Qubit hat einen besonderen Anwendungszweck und wird im weiteren als Borrow-Qubit bezeichnet.
Insgesamt besteht das Zielregister also aus \(n+1\) Qubits wenn der Modulus \(N\) einer Größenordnung von \(2^n\) entspricht.
Alle nachhaltigen Veränderungen durch den Quantenschaltkreis betreffen ausschließlich das Zielregister.
Des Weiteren verwendet die Quantenschaltung noch ein Qubit, welches initial im Zustand \(\ket{0}\) beginnt.
Der Zustand dieses einzelnen Qubits bedingt eine Fallunterscheidung in der Berechnung des Quantenschaltkreises und 
wird daher im Weiteren als Bedingungs-Qubit bezeichnet. 

Um die Berechnung des Quantenschaltkreises nachvollziehen zu können, 
wird die Auswirkung der verwendeten Gatter im einzelnen erklärt.
Die erste Quanten-Addition sorgt dafür, 
dass auf den initialen Zielregister \(\ket{\phi(b)}\) eine Addition mit \(a\) erfolgt und 
dadurch \(\ket{\phi(b + a)}\) entspricht.
Darauf folgt eine Quanten-Subtraktion die mit dem Subtrahend \(N\) auf das Zielregister mit
\(\ket{\phi(b + a - N)}\) wirkt. 
Daraus resultieren zwei mögliche Zustände:
\[1.~(b+a) \geq N~\rightarrow~\ket{\phi((b+a) - N)}_{n+1};~2.~
(b+a) < N~\rightarrow~\ket{\phi(2^{n+1}-(N-(a+b)))}_{n+1}
  \]
Im erste Fall entspricht das Ergebnis der korrekten Berechnung von \(a+b \mod N\).
In diesem Fall ist der Registerinhalt mit \(a+b \mod N\) kleiner als \(N\), 
weshalb das Borrow-Bit nicht gesetzt ist.
Im Gegensatz dazu ist das Ergebnis im zweiten Fall fehlerhaft.
Da bei \((b+a) < N\) bereits der Rest der modularen Restklasse im Register steht, 
wird durch die Quanten-Subtraktion ein \(N\) zu viel vom Registerinhalt abgezogen.
Dadurch entsteht ein underflow im Zielregister wodurch das Borrow-Bit gesetzt wird.

Als nächstes wirkt die inverse Quanten-Fourier-Transformation auf das Zielregister und führt eine Transformation in die Standartbasis durch.
Aufgrund des Basiswechsels in die Standartbasis beschreiben die Zustände der Qubits des Zielregisters nun das Ergebnis in der Binärdarstellung.
In der Binärdarstellung befindet sich im ersten Fall das Borrow-Bit im Zustand \(\ket{0}\) und im zweiten Fall im Zustand \(\ket{1}\).
Anhand dieser Unterscheidung kann man eine bedingte Operation mittels eines kontrollierten X-Gatter realisieren.
Dafür kontrolliert das Borrow-Bit ein X-Gatter welches auf das Bedingungs-Qubit wirkt.
Dadurch wird das Bedingungs-Qubit ausschließlich im zweiten Fall in den Zustand \(\ket{1}\) versetzt.
Danach wird das Zielregister wieder in die Fourier-Basis transformiert, 
indem die Quanten-Fourier-Transformation angewendet wird.
Anschließend kontrolliert das Bedingungs-Qubit eine Quanten-Addition mit dem Summanden \(N\) auf das Zielregister.
Die kontrollierte Quanten-Addition wird also nur im zweiten Fall angewendet und korrigiert das vorher ungültige Ergebnis,
zu dem korrekte:
\[
\ket{\phi(2^{n+1}-(N-(a+b)))}_{n+1} \underrightarrow{~+N~} 
\ket{\phi(2^{n+1}+(a+b))}_{n+1} \underrightarrow{\text{overflow }2^{n+1}}
\ket{\phi(a+b)}_{n+1}
  \]
Nun befinden sich in beiden Fällen das korrekt berechnete Ergebnis im Zielregister mit \(\ket{\Phi(a+b)\mod N}\).
Somit ist die Berechnung der modularen Addition abgeschlossen.

Je nach dem, welcher der beiden Fälle eingetreten ist, befindet sich das Bedingungs-Qubit in einem anderen Zustand.
Um zu verhindern, dass sich das Bedingungs-Qubit in einem ungewissen Zustand befindet und dadurch zum "`Trash"'-Qubit wird, 
wird der initiale Zustand \(\ket{0}\) des Bedingungs-Qubit durch die restlichen Gatter des Quantenschaltkreises wiederhergestellt.
Ein eindeutiger Zustand ermöglicht, dass das Bedingungs-Qubit bei weiteren Berechnungen wiederverwendet werden kann.

Um das Bedingungs-Qubit zurückzusetzen, 
wird zuerst eine Quanten-Subtraktion mit \(a\) auf das Zielregister angewendet.
Um die Auswirkungen dieser Quanten-Subtraktion hervorzuheben, wird der Registerinhalt für beide Fälle untersucht:
Im ersten Fall war \((b+a) \geq N\) weswegen die Subtraktion von \(N\) zum Ergebnis der modularen Addition führte.
Da \(a, b < N\) gilt, ist das Ergebnis der modularen Addition kleiner als \(a\).
Die Quanten-Subtraktion mit \(a\) führt also zu einem Underflow, wodurch das Borrow-Bit gesetzt wird.
Im zweiten Fall, war \((b+a) < N\).
Deswegen wurde die Subtraktion von \(N\) nicht durchgeführt, 
beziehungsweise wieder rückgängig gemacht, da \((b+a)\) bereits das Ergebnis der modularen Addition ist.
Für den zweiten Fall gilt als Ergebnis also \((b+a)\) und dies ist größer als \(a\), 
darum kommt es zu keinem Underflow und das Borrow-Bit ist deswegen nicht gesetzt.
Der Zustand des Borrow-Bits ist nun also im Vergleich zu den Fällen des vorherigen Abschnitts der Quantenschaltung invertiert.

Um das Borrow-Bit auslesen zu können, 
wird analog zum vorherigen Abschnitt des Quantenschaltkreises eine inverse Quanten-Fourier-Transformation durchgeführt.
Anschließend wird ein X-Gatter auf das Borrow-Bit angewendet.
Dadurch befindet sich das Borrow-Bit nun in denselben Zuständen wie in den beiden Fällen des ersten Abschnitts der Quantenschaltung.
Im nächsten Schritt kontrolliert das Borrow-Bit ein X-Gatter, das auf das Bedingungs-Qubit wirkt.
Dadurch wird es wieder in den initialen Zustand \(\ket{0}\) versetzt.

Um den Zustand \(\ket{\Phi(a+b)\mod N}\) des Zielregister wiederherzustellen, 
werden die Gatter bis zur Quanten-Subtraktion mit \(a\) inverse angewendet.
Zunächst wird ein selbstinverses X-Gatter auf das Borrow-Bit angewendet.
Danach folgt die Quanten-Fourier-Transformation, 
die die Inverse der inversen Quanten-Fourier-Transformation darstellt.
Abschließend folgt die Quanten-Addition mit \(a\), 
um die zuvor erfolgte Quanten-Subtraktion von \(a\) auf das Zielregister zu revertieren.

Das Zielregisters beinhaltet nun also den gewünschten Zustand \(\ket{\Phi(a+b)\mod N}\) für \(a, b < N\).
Des weiteren befindet sich das Bedingungs-Qubit wieder im initialen Zustand und kann für weitere Rechnungen wiederverwendet werden.

\vspace{1em}

Wenn man den Quantenschaltkreis in Abbildung~\ref{fig:modulare_addition_paper} betrachtet,
fällt auf, fällt auf, 
dass nicht alle Quanten-Additionen und Quanten-Subtraktionen kontrolliert durchgeführt werden.
Die modulare Addition soll nur dann ausgeführt werden, wenn die Kontroll-Qubits \(\ket{c_1}\) und \(\ket{c_2}\) gesetzt sind.
Dies ist zurückzuführen auf die Bedingung \(b < N\), 
die dafür sorgt, dass der Rest des Quantenschaltkreises lediglich die Identitätstransformation durchführt, 
wenn die Kontroll-Qubits nicht aktiviert sind~\cite{beauregard2003circuit}.
Der Grund, weshalb nicht alle Quanten-Additionen, 
Quanten-Subtraktion und gegebenenfalls sogar die (inversen) Quanten-Fourier-Transformationen kontrolliert sind,
liegt in der erhöhten Komplexität, die dadurch im Quantenschaltkreis entstehen würde~\cite{beauregard2003circuit}.
Wenn ein Kontroll-Qubit nicht aktiviert ist, wird das kontrollierte Gatter zwar ausgeführt, jedoch ohne praktische Wirkung.
Abbildung~\ref{fig:gatedef_U1U2U3_CNOT} zeigt die physikalische Implementierung dreier Single-Qubit-Gatter im Vergleich zu einem kontrollierten X-Gatter.
Wie aus der Abbildung erkennbar, 
benötigt das kontrollierte Gatter mehr Hardware-Elemente als die drei Single-Qubit-Gatter.
Somit ist es nicht möglich die Komplexität des Quantenschaltkreis zu verringern, 
indem zusätzliche Gatter kontrolliert angewendet werden.
\begin{figure} [H]
  \caption{Physikalische Implementierung~\cite{ibmqx5}}
  \label{fig:gatedef_U1U2U3_CNOT}
  \includegraphics[width=\columnwidth,height=10cm]{gatedef_U1U2U3_CNOT.png}
  \centering
  \end{figure}

Ein weiterer Aspekt, 
der die Komplexität der Quantenschaltung für die modulare Addition erhöht, 
ist der Abschnitt, der das Bedingungs-Qubit zurücksetzt.
Wäre es möglich, diesen Teil auszulassen, könnten zwei der insgesamt fünf Quanten-Additionen bzw. 
Quanten-Subtraktionen sowie die Hälfte der (inversen) Quanten-Fourier-Transformationen eingespart werden.

Es gibt die Möglichkeit, bei einem Qubit einen Reset durchzuführen, wodurch dieses den Zustand \(\ket{0}\) annimmt.
In Qiskit kann dies mit der Funktion \texttt(Reset)~\cite{qiskitReset} realisiert werden.
Die Verwendung ist jedoch nicht zielführend, da diese durch keine unitäre Abbildungsmatrix
beschrieben werden kann.
Somit ist die Reset Transformation nicht unitär und würde deswegen dazu führen, dass alle Quantenalgorithmen, 
die diese Funktion nutzen, ebenfalls nicht mehr unitär wären.
Da die Quanten-Phase-Estimation den Eigenwert einer unitären Transformation extrahiert, 
ist diese Funktion für die Anwendung im Shor-Algorithmus ungeeignet.

\begin{figure} [H]
  \caption{Qiskit modulare Addition}
  \label{fig:ModularAddition}
  \includegraphics[width=\columnwidth]{modulareAddition.png}
  \centering
  \end{figure}

Abbildung~\ref{fig:ModularAddition} zeigt die Implementierung in Qiskit, 
die durch den Code der Funktion \texttt{Modular\_Adder\_Gate}~\ref{code:ModularAddition} beschrieben wird.
Die Funktion \texttt{Modular\_Adder\_Gate} erwartet als Parameter den Summand \(a\) und die Zahl \(N\) jeweils als eine 
Liste, welche die Zahl in Binärdarstellung beschreibt.
Dabei repräsentiert der erste Index beider Listen das Least-Significant-Bit.
Wenn \(N\) der Größenordnung \(2^n\) entspricht, 
müssen die Listen jeweils von der Länge \(n+1\) sein wobei das Most-Significant-Bit der \(0\) entspricht.
Die Funktion definiert drei Wertebereiche, die die Positionen der Qubits beschreiben.
\textit{c\_qbits} definiert die beiden Kontroll-Qubits, \textit{b\_qbits} die Qubits, welche den Summanden  \(b\) 
in der Fourier-Basis enthalten und textit{cond\_qbit} repräsentiert die Position des Bedingungs-Qubit.
Dadurch das \textit{b\_qbits} genau so groß definiert wird, 
wie die Liste des Summand \(a\) lang ist, wird sichergestellt, 
dass das \textit{b\_qbits} Register ein extra Qubit enthält für das Borrow-Bit.


 

\begin{figure}[H]
  \caption{Modulare Addition in Qiskit}
  \label{code:ModularAddition}
\begin{minted}[linenos]{python}    
def Modular_Adder_Gate(a_bin: list[int],N_bin: list[int]) -> qiskit.circuit.gate:
  c_qbits = [0,1]
  b_qbits = list(range(2, len(a_bin)+2))
  cond_qbit = len(a_bin)+2
  m_a_g = qiskit.QuantumCircuit(2 + len(a_bin) + 1) 
  m_a_g.append(A_Gate(a_bin).control(2), c_qbits + b_qbits)
  m_a_g.append(S_Gate(N_bin),b_qbits)
  m_a_g.append(QFT_Gate(len(a_bin),inverse = True, MSB_first = False), b_qbits)
  m_a_g.cnot(b_qbits[-1],cond_qbit)
  m_a_g.append(QFT_Gate(len(a_bin),inverse = False, MSB_first = False), b_qbits)
  m_a_g.append(A_Gate(N_bin).control(1), [cond_qbit] + b_qbits)
  m_a_g.append(S_Gate(a_bin).control(2), c_qbits + b_qbits)
  m_a_g.append(QFT_Gate(len(a_bin),inverse = True, MSB_first = False), b_qbits)
  m_a_g.x(b_qbits[-1])
  m_a_g.cnot(b_qbits[-1],cond_qbit)
  m_a_g.x(b_qbits[-1])
  m_a_g.append(QFT_Gate(len(a_bin),inverse = False, MSB_first = False), b_qbits)
  m_a_g.append(A_Gate(a_bin).control(2), c_qbits + b_qbits)
  m_a_g = m_a_g.to_gate()
  m_a_g.name = "Add " + str(binToDez(a_bin)) + " Mod " + str(binToDez(N_bin))
  return m_a_g
  \end{minted}
\end{figure}





















\section{Resultate}
In der nachfolgenden Analyse liegt der Fokus auf der Evaluierung der implementierten Varianten des Shor-Algorithmus hinsichtlich ihres Ressourcenverbrauchs und 
der Laufzeitkomplexität. 
Angesichts der signifikanten Auswirkungen des Algorithmus für die Integrität des kryptographischen Systems des RSA-Verfahrens  
orientieren sich die zugrunde gelegten Bewertungskriterien an den Richtlinien für Post-Quantenkryptographie des \textit{National Institute of Standards and Technology}. 
Die erzielten Resultate ermöglichen eine Einordnung in die Größenordnung der erforderlichen Laufzeit und 
erlauben einen direkten Vergleich mit den Ergebnissen anderer Arbeiten.

\subsection*{Ressourcenbedarf und Laufzeit}
In der Publikation "'Submission Requirements and Evaluation Criteria for the Post-Quantum Cryptography Standardization Process"`~\cite{NISTPQC} des \textit{National Institute of Standards and Technology}
werden drei Sicherheitsstufen definiert, die sich auf die Laufzeitanforderungen von Quantencomputern in Bezug auf kryptographische Angriffe beziehen.

Als Metrik für die Laufzeitevaluation wird der Parameter \textit{MAXDEPTH} eingeführt, 
welcher die Anzahl der Quantengatter für die längste serielle Ausführung eines Quantenalgorithmus erfasst.
Die Auswahl dieser Metrik wird in der Publikation~\cite{NISTPQC} damit begründet, dass lange serielle Berechnungen Herausforderungen mit sich bringen.

Die unterste der drei Sicherheitsstufe beginnt bei einer MAXDEPTH von \(2^{40}\) Quantengattern. 
Dieser Wert dient als approximative Messgröße für die Anzahl an seriell ausgeführten Quantengattern, 
die nach aktueller Prognose von Quantencomputern innerhalb eines Jahres realisiert werden können.

Die zweite Sicherheitsstufe setzt eine \textit{MAXDEPTH} von \(2^{64}\) Quantengattern an.
Dieser Wert stellt die approximative Anzahl an seriell ausgeführten Quantengattern dar, 
die nach der Prognose innerhalb eines Jahrzehnts durchgeführt werden können.

Die Obergrenze ist bei einer \textit{MAXDEPTH} von \(2^{96}\) Quantengattern festgelegt. 
Selbst unter idealen Bedingungen, 
bei denen Qubits auf atomarer Skala arbeiten und die Übertragungsgeschwindigkeit der Lichtgeschwindigkeit entspricht, 
würde diese Anzahl an Quantengattern ein Jahrtausend an Berechnungszeit erfordern.

\vspace{1em}

Als Metriken für den Ressourcenverbrauch werden die benötigte Anzahl an Qubits sowie die Gesamtanzahl an Quantengattern herangezogen. 
Die Anzahl der Qubits stellt eine notwendige Voraussetzung dar, 
die ein Quantencomputer erfüllen muss, da andernfalls die Ausführung des Quantenalgorithmus nicht möglich ist. 
Die Gesamtanzahl an Quantengattern dient als eine Kennzahl, 
die potenzielle Rückschlüsse auf die Fehlerrate durch Dekohärenz zulässt. 
Darüber hinaus finden sowohl die Anzahl an Qubits als auch die Gesamtanzahl an Quantengattern häufig Anwendung als Kennzahl in der wissenschaftlichen Literatur. 
Diese beiden Metriken ermöglichen somit einen Vergleich mit den Ergebnissen anderer Arbeiten.
\section{Verlauf}
\subsection{Rückblick} 
\subsection{Ausblick}

\appendix
\newpage
\section{Anhang}
Dieser Bachelorarbeit liegt ein CD bei. 
Der Inhalt der CD umfasst die Quantenalgorithmen, 
die in dieser Arbeit implementiert wurden, sowie den klassischen Algorithmus zur Nachberechnung. 
Zu den Quantenalgorithmen gehören die Quanten-Fourier-Transformation und 
der Shor-Algorithmus in seinen vier Varianten. 
Des Weiteren enthält die CD das Endprodukt in Form des Faktorisierungsalgorithmus. 

Alle Implementierungen sind ausführlich dokumentiert.


\printbibliography[title=Literaturverzeichnis]

\input{Chapters/Erklaerung.tex}

\end{document}
