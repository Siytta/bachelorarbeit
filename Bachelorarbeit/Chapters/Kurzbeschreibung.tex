
\section*{Kurzfassung}
In den vergangenen Jahren haben sich in der Entwicklung von Quantencomputern kontinuierliche Fortschritte ergeben. 
Aufgrund dieser Fortschritte erwarten Prognosen, 
dass in absehbarer Zukunft Quantencomputer mit kryptografisch relevanter Rechenleistung existieren könnten. 
Derzeit existieren bereits konzeptuelle Quantenalgorithmen, die in der Lage sind, 
bestimmte mathematische Probleme effizient zu lösen. 
Diese mathematischen Probleme spielen aufgrund ihrer Komplexität eine grundlegende Rolle in bestimmten Kryptosystemen. 
Da bisher keine Quantencomputer mit ausreichender Rechenleistung existieren, 
gibt es auch wenige konkrete Implementierungen dieser Quantenalgorithmen.

In dieser Arbeit wird der \textit{Shor-Algorithmus} implementiert. 
Mithilfe dieses Quantenalgorithmus ist es möglich, 
die Primfaktorzerlegung mit einem vertretbaren Laufzeitaufwand zu berechnen. 
Konventionelle klassische Algorithmen zeigen bei dieser Berechnung eine exponentielle Laufzeitsteigerung, 
wobei das Wachstum von der Größe der Zahl abhängt. 
Das RSA-Kryptosystem basiert auf der Schwierigkeit der Primfaktorzerlegung und 
wird dementsprechend durch den \textit{Shor-Algorithmus} bedroht. 

Die in dieser Arbeit durchgeführte Implementierung des \textit{Shor-Algorithmus} 
wird um mehrere Optimierungen erweitert, 
die in unterschiedlichen Kombinationen verschiedene Varianten des Quantenalgorithmus ergeben.  
Des Weiteren wird ein klassischer Algorithmus zur Nachberechnung entwickelt, 
der die Erfolgsrate des Quantenalgorithmus erhöht. 
In einer Analyse werden die Laufzeiten und 
der Ressourcenbedarf der implementierten Varianten anhand von Kriterien für kryptografische Angriffe durch Quantenalgorithmen bewertet. 
Abschließend erfolgt ein Vergleich mit den Resultaten vergleichbarer Arbeiten, 
gefolgt von einer Bewertung der erzielten Resultate.

Die Implementierung findet auf der Abstraktionsebene von Quantenschaltkreisen statt. 
Dazu wird das Open-Source-Softwareentwicklungskit Qiskit verwendet.


\textbf{Schlagwörter:} \textit{Shor-Algorithus},\textit{Quanten-Computing}, \textit{Qiskit}, \textit{RSA}, \textit{Primfaktorzerlegung}

\section*{Abstract}

In recent years, there have been continuous advancements in the development of quantum computers. 
Due to these advancements, forecasts suggest that in the foreseeable future, 
quantum computers with cryptographically relevant computing power could exist. 
Currently, conceptual quantum algorithms already exist, capable of efficiently solving certain mathematical problems. 
These mathematical problems play a fundamental role in specific cryptographic systems due to their complexity. 
However, as of now, 
there are few concrete implementations of these quantum algorithms because quantum computers with sufficient computing power do not yet exist.

In this work, the Shor algorithm is implemented. With the help of this quantum algorithm, 
it is possible to compute the prime factorization with a reasonable runtime. 
Conventional classical algorithms exhibit an exponential increase in runtime for this computation, 
depending on the size of the number. 
The RSA cryptosystem relies on the difficulty of prime factorization and is, 
therefore, threatened by the Shor algorithm.

The implementation of the Shor algorithm in this work is extended with several optimizations, 
resulting in various variants of the quantum algorithm in different combinations. 
Furthermore, a classical post-processing algorithm is developed to increase the success rate of the quantum algorithm. 
In an analysis, 
the runtimes and resource requirements of the implemented variants are evaluated based on criteria for cryptographic attacks using quantum algorithms. 
Finally, a comparison is made with the results of similar works, followed by an assessment of the achieved results.

The implementation takes place at the abstraction level of quantum circuits, using the open-source software development kit Qiskit.

\textbf{Keywords:} \textit{Shor-Algorithm},\textit{Quantum-Computing}, \textit{Qiskit}, \textit{RSA}, \textit{Prime factorization}