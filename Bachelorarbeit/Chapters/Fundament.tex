\section{Fundament}
\subsection{Literatur} 
\subsubsection*{Shor's Algorithmus}
Der Algorithmus wurde erstmals in der Publikation \textit{"`Algorithms for Quantum Computation: Discrete Logarithms and Factoring"'} von Peter W. Shor veröffentlicht.

Die Arbeit von Shor umfasste zwei Algorithmen.
Der erste Algorithmus ermöglicht die effiziente Berechnung der Primfaktorzerlegung, 
während der zweite Algorithmus die effiziente Berechnung des diskreten Logarithmus ermöglicht.
Aufgrund ihrer konzeptionellen Ähnlichkeiten werden beide Algorithmen häufig kollektiv als "`Shor's Algorithmus"' bezeichnet.

Die Quantenberechnungen beider Algorithmen basieren auf arithmetische Operationen in Restklassen und der Quanten-Fourier-Transformation.
Da Shor die Umsetzung von arithmetischen Operationen in Restklassen sowie die Quanten-Fourier-Transformation nicht explizit behandelt,
stützt sich die Implementierung auf den Ergebnissen weiterer Arbeiten. 
Diese untersuchen insbesondere effiziente Methoden zur Durchführung von arithmetischer Operationen in Restklassen innerhalb eines Quantenschaltkreises.
Einige dieser Arbeiten untersuchen die Quanten-Fourier-Transformation, 
da diese ein notwendiges Element für gewisse Berechnungen darstellt.

In der Publikation \textit{"`Quantum Networks for Elementary Arithmetic Operations"'} erklären Vlatko Vedral,  Adriano Barenco und Artur Ekert,
wie die modulare Exponentiation in einem Quantenschaltkreis berechnet werden kann. 

St\'{e}phane Beauregard baut auf den Erkenntnissen von Vedral, Barenco und Ekert auf und
verbessert den Quantenschaltkreis für die arithmetische Operation der modularen Exponentiation.
Hierfür ersetzt Beauregard einen Teil des ursprünglichen Quantenschaltkreis,
der in der modularen Exponentiation für die Berechnung der Addition genutzt wurde, 
durch einen effizienteren Quantenschaltkreis,
wie ihn Thomas G. Draper in \textit{"`Addition on a Quantum Computer"'} beschreibt.
Des weiteren verwendet Beauregard diese Optimierungen in seiner Arbeit \textit{"`Circuit for Shor’s algorithm using 2n+3 qubits"'},
um eine Realisierung von Shor's Algorithmus zur Faktorisierung zu beschreiben.

Andere Arbeiten, die sich mit der Implementierung von Shors's Algorithmus auseinandersetzen,
realisieren die modularen Exponentiation, 
indem klassische Schaltkreise identisch in den Quantenkontext übersetzt werden~\cite{Vedral_1996}.
Der Nachbau von klassischen Operationen ist aufgrund der unitären Natur von Quantenoperationen nicht die effizienteste Realisierung.
Des Weiteren meiden andere Arbeiten die Realisierung der modularen Exponentiation für allgemeine Eingaben~\cite{9376169,9686492}. 
Ohne modulare Exponentiation sind diese Varianten nicht in der Lage die Berechnung variabler Werte zu verarbeiten.
Stattdessen dienen diese ausschließlich der Demonstration der Funktionsweise von Shor's Algorithmus und
sind nur in der Lage, ausgewählte Eingaben zu verarbeiten.

Bei Recherchen wurde keine allgemeine Variante gefunden,
die weniger Qubits benötigt als die Variante aus der Arbeit \textit{"`Circuit for Shor’s algorithm using 2n+3 qubits"'},
deswegen bildet diese die Grundlage der Implementierung ab.
















