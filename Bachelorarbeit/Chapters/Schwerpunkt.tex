\section{Schwerpunkt}
\subsection{Entschlüsselungsfunktion für RSA-Verfahren} 
Der Fokus dieser Arbeit liegt auf der Implementierung einer Entschlüsselungsfunktion, die in der Lage ist, 
den privaten Schlüssel des RSA-Verfahrens aus dem zugehörigen öffentlichen Schlüssel abzuleiten.

Das RSA-Verfahren stellt ein sogenanntes asymmetrisches Kryptosystem dar.
Bei asymmetrischen Kryptosystemen kommt ein mathematisch verknüpftes Schlüsselpaar zum Einsatz. 
Dieses Schlüsselpaar besteht aus einem öffentlichen und einem privaten Schlüssel.
Wobei der privaten Schlüssel ausschließlich dem Eigentümer des Schlüsselpaares zugänglich sein sollte.
Mit dem privaten Schlüssel ist der Eigentümer in der Lage, Nachrichten zu signieren. 
Ein Nutzer kann unter der Verwendung des öffentlichen Schlüssels die Authentizität der signierten Nachricht überprüfen. 
Hingegen erlaubt der öffentliche Schlüssel die Verschlüsselung von Nachrichten, 
deren anschließende Entschlüsselung ausschließlich mit dem privaten Schlüssel durchführbar ist.
Eine Bedingung asymmetrischer Kryptosysteme ist, 
dass die Ableitung des privaten Schlüssels aus dem öffentlichen Schlüssel eine Herausforderung von erheblicher Komplexität darstellt~\cite{1055638}. 
Um dieser Anforderung gerecht zu werden, basieren asymmetrische Kryptosysteme auf mathematischen Problemstellungen, 
deren Lösung sogar unter Nutzung eines Computers von erheblicher Komplexität ist.

Die Sicherheit des kryptografischen Verfahren RSA beruht auf der Annahme,
dass die Faktorisierung eines Produkts bestehend aus zwei großen Primzahlen in keiner vertretbaren Zeit berechenbar ist.
Andernfalls wäre es möglich, aus dem öffentlichen Schlüssel die beiden Primfaktoren zu extrahieren, 
um anschließend damit den privaten Schlüssel zu bestimmen.
Daraus ergibt sich die Folgerung, dass, sofern die Faktorisierung großer Zahlen mit geringem Aufwand bewältigt werden kann, 
das RSA-Verfahren als kompromittiert angesehen werden muss~\cite{Cormen2009}.

Bislang ist kein klassischer Algorithmus bekannt, der die Primfaktorzerlegung effizient berechnen kann~\cite{Hoever2022Krypto}.
In diesem Zusammenhang bedeutet "'effizient"', dass die Laufzeit des Algorithmus in einem höchstens polynomialen Verhältnis zur Größe der Eingabezahl anwächst.

Im Kontext der Entschlüsselungsfunktion für RSA wird eine Funktion implementiert,
die in der Lage ist, die Primfaktorzerlegung effizient zu berechnen.
Die Entschlüsselungsfunktion besteht aus einem Quantenalgorithmus und einem klassischen Algorithmus.
Der eingesetzte Quantenalgorithmus kann effektiv die Ordnung eines Elements in einer Gruppe bestimmen. 
Unter Einbeziehung der zuvor bestimmten Ordnung werden anschließend die Primfaktoren des öffentlichen Schlüssels mithilfe eines klassischen Algorithmus berechnet~\cite{Shor_1997}.
In einem abschließenden Schritt wird der private Schlüssel auf der Grundlage der berechneten Primfaktoren ermittelt.





