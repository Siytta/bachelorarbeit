\section{Resultate}
In der nachfolgenden Analyse liegt der Fokus auf der Evaluierung der implementierten Varianten des Shor-Algorithmus hinsichtlich ihres Ressourcenverbrauchs und 
der Laufzeitkomplexität. 
Angesichts der signifikanten Auswirkungen des Algorithmus für die Integrität des kryptographischen Systems des RSA-Verfahrens  
orientieren sich die zugrunde gelegten Bewertungskriterien an den Richtlinien für Post-Quantenkryptographie des \textit{National Institute of Standards and Technology}. 
Die erzielten Resultate ermöglichen eine Einordnung in die Größenordnung der erforderlichen Laufzeit und 
erlauben einen direkten Vergleich mit den Ergebnissen anderer Arbeiten.

\subsection*{Ressourcenbedarf und Laufzeit}
In der Publikation "'Submission Requirements and Evaluation Criteria for the Post-Quantum Cryptography Standardization Process"`~\cite{NISTPQC} des \textit{National Institute of Standards and Technology}
werden drei Sicherheitsstufen definiert, die sich auf die Laufzeitanforderungen von Quantencomputern in Bezug auf kryptographische Angriffe beziehen.

Als Metrik für die Laufzeitevaluation wird der Parameter \textit{MAXDEPTH} eingeführt, 
welcher die Anzahl der Quantengatter für die längste serielle Ausführung eines Quantenalgorithmus erfasst.
Die Auswahl dieser Metrik wird in der Publikation~\cite{NISTPQC} damit begründet, dass lange serielle Berechnungen Herausforderungen mit sich bringen.

Die unterste der drei Sicherheitsstufe beginnt bei einer MAXDEPTH von \(2^{40}\) Quantengattern. 
Dieser Wert dient als approximative Messgröße für die Anzahl an seriell ausgeführten Quantengattern, 
die nach aktueller Prognose von Quantencomputern innerhalb eines Jahres realisiert werden können.

Die zweite Sicherheitsstufe setzt eine \textit{MAXDEPTH} von \(2^{64}\) Quantengattern an.
Dieser Wert stellt die approximative Anzahl an seriell ausgeführten Quantengattern dar, 
die nach der Prognose innerhalb eines Jahrzehnts durchgeführt werden können.

Die Obergrenze ist bei einer \textit{MAXDEPTH} von \(2^{96}\) Quantengattern festgelegt. 
Selbst unter idealen Bedingungen, 
bei denen Qubits auf atomarer Skala arbeiten und die Übertragungsgeschwindigkeit der Lichtgeschwindigkeit entspricht, 
würde diese Anzahl an Quantengattern ein Jahrtausend an Berechnungszeit erfordern.

\vspace{1em}

Als Metriken für den Ressourcenverbrauch werden die benötigte Anzahl an Qubits sowie die Gesamtanzahl an Quantengattern herangezogen. 
Die Anzahl der Qubits stellt eine notwendige Voraussetzung dar, 
die ein Quantencomputer erfüllen muss, da andernfalls die Ausführung des Quantenalgorithmus nicht möglich ist. 
Die Gesamtanzahl an Quantengattern dient als eine Kennzahl, 
die potenzielle Rückschlüsse auf die Fehlerrate durch Dekohärenz zulässt. 
Darüber hinaus finden sowohl die Anzahl an Qubits als auch die Gesamtanzahl an Quantengattern häufig Anwendung als Kennzahl in der wissenschaftlichen Literatur. 
Diese beiden Metriken ermöglichen somit einen Vergleich mit den Ergebnissen anderer Arbeiten.