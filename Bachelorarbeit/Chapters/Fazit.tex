\section{Fazit und Ausblick}
\subsection*{Fazit}
Mit dem Ziel, eine Bibliothek von Quantenalgorithmen für die Kryptoanalyse zu entwickeln, 
wurde der Shor-Algorithmus in vier verschiedenen Varianten implementiert. 
Im Verlauf der Arbeit wurde zuerst die Funktionsweise der benötigten Bausteine, 
wie die Quanten-Fourier-Transformation und die Quantum-Phasen-Estimation, erläutert. 
Auf dieser Grundlage wurde die Verbindung zum Shor-Algorithmus hergestellt, 
der in seiner Essenz eine Form der Quantum-Phase-Estimation mit spezifischen unitären Gattern darstellt. 
Im weiteren Verlauf der Arbeit wurde die Funktionsweise und die Implementierung der spezifischen unitären Gatter 
sowie des gesamten Shor-Algorithmus vorgeführt. 

Aufbauend auf der resultierenden Struktur des Shor-Algorithmus wurden zwei Optimierungen angewendet, 
die den Bedarf an Qubits sowie den gesamten Bedarf an Quantengattern signifikant verringern. 
Bei der Untersuchung einer dieser Optimierungen wurde festgestellt, 
dass es einen Fehler in einer der verwendeten Quellen gibt. 
Unter Hinzunahme von Simulationen wurde nachgewiesen, 
dass diese Unstimmigkeit zu einem fehlerhaften Quantenschaltkreis führt.

Des Weiteren wurde ein klassischer Algorithmus entwickelt, 
um die Erfolgsrate des Shor-Algorithmus zu steigern. 
Dieser Algorithmus führt Berechnungen basierend auf den Messergebnissen des Shor-Algorithmus durch und wurde speziell entworfen, 
um auf drei mögliche Sonderfälle eines Messergebnisses zu reagieren. Diese Sonderfälle wurden anhand von Simulationen erläutert.

In einer abschließenden Analyse wurden die vier resultierenden Varianten des Shor-Algorithmus, 
die im Prinzip aus unterschiedlichen Kombinationen der Optimierungen bestehen, 
hinsichtlich ihrer Laufzeit und ihres Ressourcenbedarfs analysiert. 
Die Laufzeitanalyse orientierte sich an den Kriterien des \textit{National Institute of Standards and Technology}, 
die realistische Laufzeiten für Quantenalgorithmen zu kryptoanalytischen Zwecken vorgeben.
Das Resultat dieser Analyse hatte das eindeutige Ergebnis, 
dass von den vier Varianten die implementierte \textit{Iterative-Approximative} Variante die effizienteste darstellt.

Des Weiteren wurde die Konstruktion der verwendeten Bausteine in Bezug auf die Laufzeit und den Gesamtbedarf an Ressourcen im Vergleich zu ähnlichen Arbeiten analysiert. 
Auch hier ergab sich das Ergebnis, 
dass die \textit{Iterative-Approximative} Variante sowohl eine kürzere Laufzeit als auch einen geringeren Bedarf an Qubits aufweist, 
was sie effizienter macht als die Bauart aus der vergleichbaren Arbeit.

Um die Vorteile des klassischen Algorithmus zur Nachberechnung zu bewerten, 
wurde die durchschnittliche Anzahl der benötigten Messdurchläufe berechnet und 
mit einem Ansatz ohne klassische Nachberechnung verglichen. 
Das Ergebnis dieses Vergleichs führte zu der eindeutigen Erkenntnis, 
dass die Verwendung der Nachberechnung die Anzahl der benötigten Durchläufe des Quantenalgorithmus effektiv reduzieren kann. 

Abhängig von den Kriterien des \textit{National Institute of Standards and Technology} wurde festgestellt,
dass \emph{keine} der implementierten Varianten effiziente Möglichkeiten zur Durchführung eines kryptografischen Angriffs bei den derzeit empfohlenen RSA-Schlüssellängen bieten.
Für RSA-Verfahren mit kürzeren Schlüssellängen als den derzeit empfohlenen, die immer noch weit verbreitet sind,
erfüllen jedoch alle vier Varianten die Kriterien und
stellen somit effiziente Quantenalgorithmen zur Kryptoanalyse dar.

\subsection*{Ausblick}

Die Ergebnisse dieser Arbeit zeigen, 
dass der Fokus zukünftige Untersuchungen des Shor-Algorithmus für kryptoanalytische Zwecke, 
auf der Reduktion der benötigten Laufzeit des Quantenalgorithmus liegen sollte. 
Es ist naheliegend, dass dafür eine neue Konstruktionsart der spezifischen unitären Gatter nötig ist. 

Eine weitere potentielle Möglichkeiten besteht darin, 
den Quantenalgorithmus auf Kosten von zusätzlichen Qubits zu parallelisieren. 
Eine solche Untersuchung sollte die Abwägung zwischen zusätzlichen Qubits und eingesparter Laufzeit näher beleuchten.

\bigskip

Des Weiteren existiert noch das komplexe Themenfeld der Korrektur von Fehlern, 
die durch Dekohärenz erzeugt werden. 
Aufgrund der Komplexität dieses Themas, 
sollte für die Bearbeitung ein fortgeschrittener Kenntnisstand im Quanten-Computing vorhanden sein.
Zusätzlich sollte vorab eine gründliche Untersuchung der geplanten Quantencomputer stattfinden, 
um so konkreteres über prognostizierte Fehlerraten zu erfahren. 
Es ist möglich, dass für die zukünftigen Quantencomputer Architekturen geplant sind, 
bei denen eine Fehlerkorrektur auf der Abstraktionsebene von Quantengattern obsolet ist.

\clearpage